%% Before beginning to type your dissertation, download and READ 
%% the Graduate School's formatting guide, which can be found at 
%% http://grad.msu.edu/etd
%% and clicking Formatting Guide in the left hand column.
%% Also get the latest version of  msuphddissertation.cls and the template file
%% at http://www.math.msu.edu/~weil/MSU_Ph.D._Dissertation.zip
%% Send questions to weil@math.msu.edu

%%%%%%%%%%%%%%%%%%%%%%%%%%%%
%%%%%%%%  NOTE   %%%%%%%%%%%%%%
%% PREPARING A DISSERTATION WITH THIS CLASS FILE DOES NOT %%%
%% GUARANTEE THAT THE GRADUATE SCHOOL WILL APPROVE IT. %%%
%%%%%%%%%%%%%%%%%%%%%%%%%%%%%%%

%% To view a video presentation of this template, visit
%%  https://www.math.msu.edu/latex/dissertation/

\documentclass{msuphddissertation}
%% This is the first command that must appear in your thesis.
%% Insert packages you wish to use except setspace, subfig
%% geometry and pdflscape. 
%% These packages are loaded automatically.
\usepackage{graphicx, amsmath, physics}
\usepackage{bm}% bold math

% These are to get an APS-like bibliography
% See: https://tex.stackexchange.com/questions/15677/revtex-4-1-bibliography-style-in-other-classes
\usepackage[sort&compress,numbers]{natbib}
\usepackage{doi}
\usepackage{hyperref}
\usepackage{tcolorbox}
\usepackage{rotating}

\newcommand{\hfb}{UNEDF1$_{\mathrm{HFB}}$}
%\newcommand{\Og}{$^{294}_{118}$Og$_{176}$}
%\newcommand{\Pb}{$^{208}_{\hphantom{2}82}$Pb$_{126}$}
%\newcommand{\Kr}{$^{86}_{36}$Kr$_{50}$}
%\newcommand{\Xe}{$^{126}_{\hphantom{1}54}$Xe$_{72}$}
%\newcommand{\Cf}{$^{254}_{\hphantom{2}98}$Cf$_{156}$}
%\newcommand{\Pt}{$^{178}_{\hphantom{1}78}$Pt$_{100}$}
%\newcommand{\Hg}{$^{180}_{\hphantom{1}80}$Hg$_{100}$}
\newcommand{\Og}{$^{294}$Og}
\newcommand{\Pb}{$^{208}$Pb}
\newcommand{\Kr}{$^{86}$Kr}
\newcommand{\Xe}{$^{126}$Xe}
\newcommand{\Cf}{$^{254}$Cf}
\newcommand{\Pt}{$^{178}$Pt}
\newcommand{\Hg}{$^{180}$Hg}

\newcommand{\MGCM}{$\mathcal{M}^{\rm G}$}
\newcommand{\MGCMp}{$\mathcal{M}^{\rm GCM}$}
\newcommand{\MATDHFp}{$\mathcal{M}^{\rm AP}$}
\newcommand{\MATDHF}{$\mathcal{M}^{\rm A}$}
\newcommand{\Mcons}{$\mathcal{M}^{\rm cst}$}
%% IMPORTANT: Load only those packages you know you will use.
%% Some packages can cause conflicts resulting in improper formatting.
\author{Zachary Matheson} %% Put your name in full as it is officially recognized by Michigan State University here.
%\title{Fission in Exotic Nuclei} %% Put the title of your dissertation here.
\title{Fission in Exotic Nuclei Using Density Functional Theory} %% Put the title of your dissertation here.
%% Go to http://grad.msu.edu/etd/docs/DegreeGrantingUnits.pdf
%% and find your GRADUATE DEGREE GRANTING UNIT/PROGRAM
%% and DEGREE.
\unit{Physics --- Doctor of Philosophy \\ Computational Mathematics, Science and Engineering --- Dual Major} %% Copy and paste these two items here 
%% separated by a dash, created by typing ---.
%% ONLY ENTRIES FROM LIST MENTIONED ARE ACCEPTABLE!!

%% Put additional preamble items here.

\begin{document}

\maketitlepage %%This command will produce the title page of your thesis.

%% If you wish to include a "public abstract" (i.e.; in layman's terms), remove the "%" 
%% in from of the command \begin{pub abstract} and remove the "%" in front of
%% \end{pub abstract} below. A public abstract isn't required, but might be useful
%% for some readers.
%\begin{pubabstract}
%%Type the text of your public abstract here. A public abstract is optional.
%\end{pubabstract}

\begin{abstract}
%% Type your abstract here. An abstract is REQUIRED and limited to two pages.
%% The abstract must not include any figures.
Historically, most experimental and theoretical studies of fission have centered around the region of actinides near $^{235}$U, which includes isotopes of uranium, plutonium, and thorium relevant for reactor physics and stockpile stewardship. Isotopes in this region tend to fission asymmetrically, with the larger prefragment influenced by the doubly-magic shell structure of $^{132}$Sn and resulting in a heavy fragment distribution centered around ${\sim}^{140}$Te. However, given the nuclear physics community's recent interest in rare and exotic nuclei, we have applied our model to study spontaneous fission primary fragment yields in exotic systems found in other regions of the nuclear chart, including neutron-deficient {\Pt}, superheavy {\Og}, and neutron-rich nuclei with relevance to the r-process such as $^{254}$Pu.
\end{abstract}

%% If you wish to have a copyright page, remove the "%" 
%% in front of \begin{copyrt}
%% and remove the "%" in front of \end{copyrt}.
%% An acceptable form of a copyright page  
%% will be generated automatically. 
%% TO INCLUDE A COPYRIGHT, YOU MUST REGISTER
%% IT. See the Formatting Guide for instructions. 
%\begin{copyrt}
%\end{copyrt}


%%% If you wish to have a dedication, remove the "%" in front of
%%% \begin{dedication} and remove the "%" in front of
%%% \end{dedication} below.
%%% A dedication must be single-spaced and 
%%% centered on the page. Both will be done automatically. 
%
%\begin{dedication} 
%% Type your dedication here. A dedication is optional.
%Dedicated to Eli
%\end{dedication}
%
%%% If you wish to have an acknowledgment, remove the "%" in front of  \begin{acknowledgment}
%%% and remove the "%" in front of  \end{acknowledgment} below.  
%\begin{acknowledgment}
%% Type your acknowledgment here. An acknowledgment is optional.
%I wish to express my gratitude to Witek, who provided me with direction and with many great and wonderful opportunities to travel and meet new people. I also express my appreciation for the cameraderie he has cultivated within his group through his weekly group meetings, regular lunch outings, and occasional house gatherings.
%
%And speaking of Witek's group, I have rubbed shoulders with many terrific young scientists, many of whom have helped me along the way. Here I recognize Erik, Kevin, Bastian, Yannen, Chunli, Jimmy, Samuel, Nicolas, Rolo, Yue, Nobuo, Futoshi, Mao, Maxwell, Mengzhe, Tong, ...
%
%Samuel, Jhilam, Nicolas
%
%I am grateful for the many graduate students I've encountered along the way. The community of graduate students at the NSCL is (one of its greatest assets?), and perhaps it's unfair of me to single out any individual. Nevertheless, there are 4(?) individuals who made specific, important contributions at critical moments of my PhD development/career. I would like to recognize:
%
%Terri, for being reliable
%Juan, for being relatable
%Wei Jia, for being empathetic
%Mao, for being genuine/my friend/sincere
%
%Other individuals I particularly wish to mention by name include Amy, John, Hao, Becky, and Chris.
%\end{acknowledgment}

%% If you wish to have a preface, remove the "%" in front of \begin{preface}
%% and remove the "%" in front of \end{preface} below. The formatting of
%% a preface isn't specified, but it is included in the TOC.
%\begin{preface}
%% Type your preface here. A preface is optional.
%\end{preface}

\TOC %% This command produces the Table of Contents. DO NOT REMOVE IT!

%% If your document contains tables, remove the "%" in front of 
%%  the following line.
%\LOT

%% If your document contains figures, remove the "%" in front of
%% the following line.
\LOF

%%%% LIST OF SYMBOLS OR LIST OF ABBREVIATIONS %%%%
%% If you wish to have a list of symbols or a list of abbreviations, 
%% it should be here. For a list of symbols remove the "%" in front of 
%% \begin{symbols} and remove the "%" in front of \end{symbols} below.
%\begin{symbols}
%% Type your list using a list environment here.
%\end{symbols}
%% Similarly for a list of abbreviations remove the "%" in front of 
%% \begin{abbrev} and remove the "%" in front of \end{abbrev} below.
%\begin{abbrev}
%% Type your list using a list environment here.
%\end{abbrev}
%% The list will be included in the TOC as
%% KEY TO SYMBOLS or KEY TO ABBREVIATIONS
%%%%%%%%%%

\newpage
\pagenumbering{arabic}
\begin{doublespace}

%% Put the body of your dissertation here. 
%% DO NOT include the bibliography or any appendices.
%% These topics will be discussed later.
\chapter{Introduction}\label{chap:Intro}

\section{History of fission theory}
Nuclear fission is the fundamental physical process by which a heavy nucleus decays to two smaller nuclei with comparable masses, and a proper understanding of fission is critical for applications in reactor physics, nuclear astrophysics, and stockpile stewardship. Fission was first observed by Hahn and Stra\ss{}mann in 1939 \cite{Hahn1939} as they bombarded uranium atoms with neutrons and detected barium, but the men could not explain their observations at the time. An explanation came shortly thereafter in letters to the editor by Meitner and Frisch \cite{Meitner1939b} and by Bohr \cite{Bohr1939a}. In Meitner and Frisch's paper they said ``On account of their close packing and strong energy exchange, the particles in a heavy nucleus would be expected to move in a collective way which has some resemblance to the movement of a liquid drop. If the movement is made sufficiently violent by adding energy, such a drop may divide itself into two smaller drops\dots It seems therefore possible that the uranium nucleus has only small stability of form, and may, after neutron capture, divide itself into two nuclei of roughly equal size (the precise ratio of sizes depending on finer structural features and perhaps partly on chance).'' A different form of fission, dubbed spontaneous fission because it occurs without bombardment by neutrons or any other projectiles, was reported by Flerov and Petrjak in a single-paragraph letter to \textit{Physical Review} in 1940 \cite{Flerov1940}. For the remainder of this dissertation, I will be referring mainly to spontaneous fission unless stated otherwise.

%It is a highly-collective process involving all the constituent nucleons of the system, and thus since its discovery, it has been described via large shape deformations of an otherwise spherical (or nearly-spherical) ``drop'' of nucleons \cite{Bohr1939}. 

Fission is relatively simple to conceptualize, but remarkably difficult to explain quantitatively. Making fission predictions based on fundamental nuclear theory is challenging because of the large number of particles involved, along with the complex collective interactions which take place when one system deforms and becomes two. Historyically, one could argue that theoretical attempts to describe nuclear fission have leapt forward in three major waves.

\subsection{Liquid drop model}
The first major wave of nuclear fission theory goes back to the very beginning of the nuclear age, with the liquid drop model in the 1930s. The liquid drop model was first developed by Weizs\"acker in 1935 \cite{Weizsacker1935} as a way of describing the collective properties of nuclei. It was later adapted by Bohr and Wheeler to quantitatively describe nuclear fission in terms of bulk properties of nuclei \cite{Bohr1939}. This model was able to successfully describe nuclear binding energies and the energetics of nuclear fission.

The liquid drop model had its weaknesses, however. For example, it could not explain the fission fragment asymmetry which characterizes spontaneous fission in many actinides. Furthermore, until the 1960s, nuclear fission was treated as though separate from the rest of nuclear physics.%.. See the first few paragraphs of \cite{Grant1976} for more of this story.

\subsection{Strutinsky shell correction approach}
The second theoretical wave came amidst a renewed interest in fission, triggered by the discovery of fission isomerism in 1962 by Polikanov, et al \cite{Polikanov1962}. This was understood as a manifestation of nuclear shape deformation based on a prediction by Nilsson \cite{Nilsson1955}. In 1967, Strutinsky added a quantum mechanical correction to the liquid drop energy in order to account for the added stability that occurs when a nucleus contains a ``magic number'' of protons and/or neutrons \cite{Strutinsky1967, Strutinsky1968, Brack1972}. These models go by the name ``microscopic-macroscopic'' because they combine the ``macroscopic'' bulk properties of the liquid-drop model with the ``microscopic'' quantum mechanical Strutinsky shell corrections.

These microscopic-macroscopic (``micmac'') fission models are computationally fairly inexpensive, and can achieve quite satisfactory results (some recent highlights from the Los Alamos and Warsaw groups include Refs. \cite{Moller2015a,Moller2015b,Jachimowicz2013,Jachimowicz2017}). However, since the model is based on a phenomenological description of what is actually a quantum many-body system, its predictive power is limited, and there is no clear way of making systematic improvements. A more reliable approach would be to consider individual nucleon states using some kind of quantum many-body method. 

\subsection{Self-consistent models and the supercomputing era}
The third major wave is taking place now, heralded by the age of supercomputers. Fission was listed in a recent 2017 report to the Department of Energy \cite{Carlson2017} as one of the problems which motivates the drive towards exascale computing. For large systems with many, many particles, density functional theory (DFT) is a way to recast the Schr\"{o}dinger equation involving $\sim$200 particles into a simpler problem involving only a few densities and currents (see Section \ref{sect:DFT}). With DFT as a way of calculating nuclear properties quantum-mechanically, one can then combine self-consistent methods with modern high-performance computing platforms to predict fission properties, such as lifetimes and fragment yields. Fortunately, a great deal of work has been done to achieve exactly this (see a recent review article on self-consistent approaches to fission in \cite{Schunck2016}). Some of the ideas which are used were inspired by lessons learned from micmac and other, simpler models; others are unique to DFT. One approach is described in detail in Chapter \ref{chap:Model}, and this is the model used to prepare this dissertation.

These advances in computing come simultaneously with advances in accelerator design and technology and other advances which allow experimental nuclear physics to reach far beyond what has been done before. For instance, the Facility for Rare Isotope Beams (FRIB) at Michigan State University is projected to be able to nearly double the number of isotopes that can be produced synthetically \cite{Baumann2016}. Together, state-of-the-art facilities for experiment and high-performance computing for theory are expected to lead to rapid advancement in our understanding of atomic nuclei and their decays.

\section{Predicting fission fragments}
Microscopic models (as self-consistent models are often called) are increasingly able to predict properties of fission fragments; however, a comprehensive description of fission fragments (including mass and fragment yields, excitation and kinetic energy distributions, angular dependence, spin, neutron emission) remains elusive in a microscopic framework. Chapter \ref{chap:Model} will discuss one approach for describing the mass and charge of fission fragments. The challenge now is to do these calculations cheaply. In every theoretical calculation, one must ask oneself ``What approximations can I safely make?'' and ``What are the important degrees of freedom for this problem?'' One may also reduce the total time-to-answer via improvements to the computational workflow itself, such as better file handling and parallelization.

Additionally, static models are not well-suited to describing the process of a single nucleus becoming two, in inherently time-dependent process. How can one precisely identify two distinct fragments when the wavefunctions of one fragment’s constituent nucleons may extend into the opposite fragment? And how do those correlations between nucleons affect the energetics of the resulting fragments? A better understanding of the mechanism of fragment formation can help guide and refine fission fragment models.


\section{Goals of the project}
Historically, most experimental and theoretical studies of fission have centered around the region of actinides near $^{235}$U, which includes isotopes of uranium, plutonium, and thorium relevant for reactor physics and stockpile stewardship. Isotopes in this region tend to fission asymmetrically, with the larger prefragment influenced by the shell structure of $^{132}$Sn and resulting in a heavy fragment distribution centered around $\sim^{140}$Te. However, recent experiments have highlighted other forms of fission which take place in other regions of the nuclear chart.

Given the aforementioned recent interest in rare and exotic nuclei, we have applied our model to study spontaneous fission in exotic systems found in other regions of the nuclear chart, with a focus on primary fragment yields. First, in Chapter \ref{chap:178Pt} we discuss bimodal fission in the neutron-deficient isotope {\Pt}, which until recently was conventionally expected to fission symmetrically. This region is a good one in which to test fission models because of the large isospin asymmetry ($N/Z\approx1.3$ in this region, compared to $N/Z\approx1.5$ near the valley of stability). Then in Chapter \ref{chap:294Og} we discuss cluster radioactivity in {\Og}, the heaviest element ever produced in a laboratory. In Chapter \ref{chap:rprocess} we move to the neutron-rich side of the nuclear chart ($N/Z>1.8$) to study isotopes which are expected to play a significant role in the astrophysical r process. %\textbf{Along the way, we will discuss some of the issues related to fragment identification and yield prediction.}

%The calculations in Chapters \ref{chap:178Pt}, \ref{chap:294Og}, and \ref{chap:rprocess} are relatively expensive. To perform large-scale exploratory studies in other regions of the nuclear chart, it will be necessary to find ways to reduce the total computational cost of these calculations. One method, still in its infancy, offers a promising approach for identifying fission fragment distributions using a significantly-reduced potential energy surface, which is by far the biggest bottleneck in our calculations. In Chapter \ref{append:Fragments}, we discuss the problem of scission and present an alternative method for identifying fragments based on the nucleon localization function.

%Alternatively, at the end of each Chapter, we say a few words about challenges faced during the project and new physical insights gained that aren't related to the overall narrative of the Chapter, but which are nevertheless useful for future model developments.

This dissertation concludes in Chapter \ref{chap:Outlook} with a discussion of results and their significance. Suggestions are then made for future model developments, computational improvements, and physical applications.
 % Introduction
\chapter{Describing Fission Using Nuclear Density Functional Theory}\label{chap:Model}
\maketitle

Since nuclei are quantum mechanical systems, they can in principle be described using the Schrodinger equation. However, in practice one finds this type of description difficult or impossible, for two reasons:

\begin{itemize}
\item In order to use the Schrodinger equation, one needs to know how to describe the interaction between particles, such as between protons and neutrons. However, protons and neutrons are made up of quarks and gluons, which interact via the strong nuclear force. Consequently, an analytic expression for the nucleon-nucleon interaction analogous to the $\frac{1}{r}$ form of the Coulomb interaction is not available. Finding different mathematical expressions which can describe the interaction between nucleons continues to be an active area of research \cite{lots of papers}
\item Even when an interaction is known, nuclei are large systems made up of many protons and neutrons. Solving the Schrodinger equation directly quickly becomes computationally intractable as the number of nucleons increases.
\end{itemize}

\subsection{Skyrme Interaction}

\subsection{Density Functional Theory}\label{sect:DFT}
Kohn-Sham - Suppose you have the density of an interacting system. There exists a unique noninteracting system with the same density
Then I believe HFB is put on top of that to do the variation part. I think I (approximately) get it now! - So just to make sure, what would DFT look like without HF/HFB? And HF/HFB without DFT?

\section{Microscopic Description of Nuclear Fission}
Should I describe these here, or in the sections when they actually get used?

\subsection{Potential Energy Surfaces}
Constrained HFB; map out many different constrained calculations and start to form a surface which resembles a topographical map. How do you decide which collective coordinates to use? Including pairing...

\subsection{WKB Approximation}

Adiabaticity: For fusion reactions, N,Z equilibrium reached in $\sim10^{-21}$ seconds, then energy/thermal equilibrium in a similar time scale, then finally mass equilibrium in $\sim10^{-19}$ - Yuri has a slide with these time scales from his talk Monday

Sort of an adiabatic approximation; useful because half-lives are long and therefore time-dependent approaches are impractical (they break down and/or become unstable or something after too many time steps, not to mention the amount of computing time). Wavefunction is assumed to be slowly-varying inside the potential barrier

Furthermore, TDHFB cannot tunnel

\subsection{Langevin Dynamics}

It has been shown (Jhilam and Nicolas' paper) that fission yields are fairly robust with respect to the dissipation strength
 % Intro to nuclear DFT
\chapter{Numerical Implementation}\label{chap:Numerical}

Since this modern era of fission theory takes advantage of high-performance computers, it is worth taking some time to discuss some of the issues which came up during the course of this research, and how modern computing tools were used to solve the problem.

\section{Calculating the PES}
By far, the most time-intensive part of a microscopic fission calculation is the calculation of the PES. For this we use a pair of DFT solvers, HFODD~\cite{Schunck2017} and HFBTHO~\cite{Perez2017}. These programs solve the HFB equations in a basis of deformed harmonic oscillators. The solver HFBTHO is limited to shapes with axial symmetry, while HFODD allows for the breaking of any symmetry needed. Broken symmetries mean that each matrix element must be computed independently, while good symmetries reduce the number of calculations which must be performed. However, since the major bottleneck of each of these programs involves constructing matrices representing Skyrme densities and currents and then diagonalizing the matrix representing the HFB Hamiltonian, this flexibility drastically increases the time-to-solution.

On the positive side, the problem of calculating a PES is embarrassingly-parallel. So while an individual point in the PES may be difficult to compute, many points can be computed simultaneously. This does have its limitations; highly-deformed configurations may be very unstable because of reasons. One fix may sometimes be to use a nearby point which converged successfully as a seed function

The procedure is performed iteratively: First, an ansatz is given for the density (either by the user or by some simple means, such as a quick Woods-Saxon calculation). Then the energy density matrix is constructed, after which it is diagonalized and a new density matrix is calculated. The procedure then repeats for a fixed number of iterations, or until a predetermined convergence criterion is satisfied In HFODD, for instance, the default convergence criterion is for the difference between the HFB energy and the total energy summed over all single-particle states to be less than some user-defined value.

%Certain parts of the procedure can be parallelized using shared-memory parallelism, such as QMULCM, which does what again? It readjusts the Lagrange parameters of the constraints based on the perturbative approximation fot he QRPA matrix.

\subsection{PES Tools}
Apart from the issue of walltime, generating a PES creates a lot of output files, which quickly becomes unwieldy. To help manage all this data, a Python module called PES Tools was created for manipulating, extracting, interpolating, and plotting PES data~\cite{PES_tools}. Aside from solver-dependent parser scripts, which collect data from the output of a DFT run and store it in the XML file format, PES Tools is not dependent on a particular DFT solver.

In particular, a submodule was created to interface between PES Tools and Fission Tools~\cite{fission_tools}, a set of codes which have relevance to fission calculations. Many of these codes are described in the following sections.

\section{Calculating the collective inertia}\label{sect:M_numerical}
The partial derivatives from equation \eqref{eq:mATDHFB-np} are computed using the Lagrange three-point formula:

\begin{equation}\label{eq:finite-diffs}
\left(\frac{\partial \mathcal{R}}{\partial q}\right)_{q=q_0} \approx 
    \frac{-\delta q'}{\delta q \left(\delta q + \delta q'\right)}\mathcal{R}(q_0-\delta q) + 
    \frac{\delta q - \delta q'}{\delta q \delta q'}\mathcal{R}(q_0) + 
    \frac{\delta q}{\delta q' \left(\delta q + \delta q'\right)}\mathcal{R}(q_0+\delta q')
\end{equation}

The accuracy and precision of the collective inertia $\mathcal{M}$ are therefore functions of the spacings $\delta q$ and $\delta q'$, and of $\mathcal{R}$. An accurate value of the collective inertia is especially important for computing half-lives, where there is an exponential dependence on $\mathcal{M}$.

\begin{figure}
	\centering
	\includegraphics[width=0.7\linewidth]{TeX_files/Num-dq_spacing}
	\caption[Collective inertia as a function of finite-difference spacing]{$\mathcal{M}_{22}$ calculated for an arbitrary configuration of $^{240}$Pu as a function of finite-difference spacing.}
	\label{fig:num-dqspacing}
\end{figure}

Figure~\ref{fig:num-dqspacing} shows the effect of different values of $\delta q = \delta q'$ on the collective inertia for an arbitrary configuration of $^{240}$Pu.

\begin{figure}
	\centering
	\includegraphics[width=0.7\linewidth]{TeX_files/Num-rho_conv}
	\caption[Norm of the difference matrix between subsequent iterations of the density]{Norm of the difference matrix between subsequent iterations of the density.}
	\label{fig:num-rhoconv}
\end{figure}

Figure~\ref{fig:num-dqspacing} shows the norm of the matrix which corresponds to the difference between the density matrix at the last iteration and the second-to-last iteration. Predictably, the norm decreases as the convergence parameter becomes tighter. This gives a sense of the uncertainty associated with the density, which in turn should be propagated through equations \eqref{eq:finite-diffs} and \eqref{eq:mATDHFB-np}.

There are additional complications which arise in the finite-temperature formalism. These are discussed in Appendix~\ref{append:TD-ATDHFB}.

\section{Minimum action path}
For the tunneling described in Section~\ref{sect:wkb}, the dynamic programming method~\cite{Baran1981} was used to minimize the action. The dynamic programming scheme proceeds inductively: Once the minimum action is known for all grid points up to a certain value of $Q_{20}$, say $q_{20}^n$, then the minimum action at each grid point in the layer with $Q_{20}=q_{20}^{n+1}$ is obtained by computing the action between each grid point in layer $n$ and each grid point in layer $n+1$, and then finding the path which minimizes the total action at each grid point in layer $n+1$:

\begin{equation}
\left.S_{min}(\vec{Q})\right|_{Q_{20}=q_{20}^{n+1}} = \mathrm{argmin}_{\vec{Q'}}\left.\left(S_{min}(\vec{Q'}) + \Delta S_{min}(\vec{Q},\vec{Q'})\right)\right|_{Q'_{20}=q_{20}^{n}, Q_{20}=q_{20}^{n+1}}.
\end{equation}

The inductive step which connects layers $n$ and $n+1$ involves several small, independent calculations which lend themselves well to shared-memory parallelism. This was implemented in the code using OpenMP, resulting in a walltime reduction from order $\mathcal{O}(n^D)$ to $\mathcal{O}(n^D/m)$, where $m$ is the number of processors (with the usual caveats that parallelization requires some additional overhead time, and that the actual speedup might be somewhat less when the number of processors $m$ reaches the same order as the number of points per layer $n^{D-1}$). In a 2D calculation, where the runtime is on the order of seconds, the difference is inconsequential. However, this speedup was essential for the analysis of a 4D PES, as described in Chapter~\ref{chap:294Og}.


\section{Langevin}
Once the action and relative probability are known for a set of points along the outer turning line, Langevin trajectories are computed originating from each outer turning point. These are straightforwardly evaluated at discrete time steps over a discretized PES mesh.

Because of the random force term in equation \eqref{eq:langevin}, a large number of trajectories per outer turning point must be computed to reduce statistical uncertainty. Fortunately, each trajectory is completely independent of every other trajectory, lending the code readily to shared-memory parallelism. However, for the Fission Tools Langevin code, distributed-memory parallelism was chosen over shared-memory parallelism in order to simplify access to shared resources, such as variables and output files. % Numerical considerations
\chapter{Cluster decay in 294Og}\label{chap:294Og}

\maketitle
\section{\label{sec:introduction}Introduction}

An exciting frontier in nuclear physics is the region of superheavy nuclei ($Z\geq104$). The latest experiments are able to push the boundaries of the nuclear chart all the way to Z=118, and new ideas are being developed to increase production and improve measurements of superheavy elements \cite{Dmitriev2016,Oganessian2016}. Due to the large number of nucleons, these nuclei push the limits of our nuclear structure models and are expected to highlight new aspects and phenomena of nuclear physics. Spontaneous fission, for example, will likely play an important role in governing the lifetimes of many of these new systems. Fission of superheavy elements may also play an important role in the astrophysical r-process, by placing an endpoint on neutron capture and starting fission cycling (see, e.g., \cite{Giuliani2017}).

As these pioneering experimental efforts are made, theory plays a critical role by guiding and interpreting the results of those experiments, as well as by filling in gaps where experiment cannot reach. However, in these exotic regions it is especially important to use only the very best and most reliable predictive models. Recently, a great deal of work has been invested in building self-consistent microscopic models of spontaneous fission which are able to predict, for instance, half-lives and fragment yields \cite{Sadhukhan2013,Sadhukhan2014,Sadhukhan2016,Sadhukhan2017}.

However, this success comes at a cost. In the adiabatic approaches that are often invoked to describe spontaneous fission, fission is described as a tunneling through a potential barrier in a multidimensional space of collective nuclear shape coordinates. Due to the large computational cost associated with calculations (their "inextinguishable thirst for computing power," as stated in \cite{Schunck2016}), this barrier is approximated using five or fewer shape coordinates in phenomenological microscopic-macroscopic models, and even fewer in mean-field approaches.

Of course, it is well-understood that some physics may be obscured in a limited collective space (see \cite{Dubray2012}). Thus, one's choice of collective coordinates is dependent on what physics are deemed important or relevant, and which aspects can be safely neglected. In mean-field models, the collective coordinates are typically chosen to be leading order terms in the shape multipole expansion: axial quadrupole moment, triaxial quadrupole moment, and axial octupole moment. Additionally, it was shown in a previous work \cite{Sadhukhan2014} that pairing correlations have a strong impact on the half-lives calculated via action minimization, and should be taken as a collective coordinate equal in importance to multipole moments or other shape-based collective coordinates.

In the following sections we try to understand the role of the collective space on fission yield predictions. In section \ref{sec:model} we describe the microscopic framework used here to calculate fragment yields, and then in section \ref{sec:results} the model is applied to the superheavy element $^{294}$Og, which is the heaviest element ever produced by humans. The paper then concludes with analysis and discussion of the results in section \ref{sec:conclusions}.

\section{$^{294}$Og}

Recent efforts to synthesize superheavy elements (SHE) have successfully produced the isotope $^{294}$Og, which has been confirmed via its alpha-decay chain. In both experiments, the researchers found evidence of alpha decay, but both also noted the possible observation of decay via spontaneous fission. This suggests the possibility that $^{294}$Og might have a similar decay time with respect to both alpha-decay and spontaneous fission.

While some authors [cite] have predicted that fission in the superheavies will proceed as with the actinides (that is, driven by the shell formation of $^{132}$Sn in one of the prefragment) our calulations predict that the dominant fission mode will be highly-asymmetric and driven by $^{208}$Pb (sometimes referred to in the literature as cluster emission).

There has been an expectation (for some reason?) that cluster emission (known also in the literature as cluster radioactivity, lead radioactivity, cluster decay, heavy-particle radioactivity, ???) might play an important role in the fission of superheavy elements, suggesting that even for such large nuclei (where the Coulomb repulsion is strong), shell structure of the prefragments still drives the determination of the fragments.

\cite{Poenaru2011, Poenaru2012} - In this paper they propose changing/extending the concept of Heavy Particle Radioactivity or Cluster Radioactivity. Also they apply some model to HPR/CR in SHE.

``A larger number of observed spontaneous fission activities enabled the establishment of a global dependency of spontaneous fission half-lives ($T_{SF}$) and the fissility of a nucleus, expressed by the ratio $Z^2/A$ which had been realized already by Seaborg [111] and also by Whitehouse and Galbraith [112]. The data, available at that time indicated for even-even nuclei an exponential dependence of the fission half-lives from $Z^2/A$. From an extrapolation of the trend it was concluded, that a nucleus will become instantaneously unstable against nuclear fission at $Z^2/A \approx 47$, which was set in correspondence with a half-life of $\approx$ 10−20 s. Interestingly, the heaviest nucleus reported to be synthesized so far, $^{294}118$ ($^{294}$Og) [65], has a value $Z^2/A \approx 47.36$. The half-life is given as $T_{1/2} =0.69+0.64−0.22$ s. Up to now four $\alpha$ decays, but no spontaneous fission was observed [65].'' - from \cite{Heßberger2017} - Og is anomolous in that it violates this extrapolated trend (as would, I am sure, most SHE).

Whether or not this PES is able to reasonably describe the CN experiments which so far have produced $^{294}$Og is uncertain, because such large compound nucleus expectation energies as appear in experiment may have quite a large effect on the topology of the PES \cite{Pei2009}

On the theory side, there have been several attempts to compute spontaneous fission half-lives and alpha-decay half-lives for many superheavy nuclei, and in many cases it is predicted that the two lifetimes will be comparable \cite{Poenaru2011, Poenaru2012, Zhang2018} [Zhang was an application of several universal CR and alpha decay models to the SHE, in order to see if the predictions, too, were universal]. These previous works have tended to rely on phenomenological models which have been tuned to smaller, more stable nuclei. Thus, it is difficult or impossible to assess these models' predictive power in the region of SHE. Thus, a goal of this work is to bring the full predictive framework of self-consistent nuclear density functional theory to bear on the problem of spontaneous fission in the SHE $^{294}$Og. This approach is relatively young in the world of nuclear fission models, but it is already producing quality results for a variety of nuclei in different regions of the nuclear chart (see, for instance, \cite{Mcdonnell2014, Sadhukhan2017, Sadhukhan2016, Tsekhanovich2018}). Some attempts in the region of SHE have already been made, using Skyrme and Gogny functionals in a 2D space \cite{Reinhard2017, Giuliani2017, Warda2012, Baran2015}.

Within these models, spontaneous fission lifetimes tend to be considerably larger than alpha decay lifetimes, ranging from $\frac{\tau_{SF}}{\tau_{\alpha}}\approx10^{-10}$ in \cite{Baran2015} and \cite{Reinhard2017} to $\frac{\tau_{SF}}{\tau_{\alpha}}\approx10^{-20}$ in \cite{Warda2012}. However, it was shown in \cite{Sadhukhan2014} that pairing correlations treated as a dynamical variable can have a substantial impact on spontaneous fission lifetimes. That is explored in the case of $^{294}$Og here.

This was done in a 4D space consisting of the coordinates $(q_{20}, q_{22}, q_{30}, \lambda_2)$

A criticism that is sometimes leveraged against self-consistent mean-field-based approaches to fission is that, due to the large computational cost associated with calculations, typically only one or two collective coordinates are used. This is in contrast to microscopic-macroscopic methods, where up to five collective coordinates are often used. Those who use SCMF methods assert that the dominant characteristics of the nuclear collective motion necessary for understanding fission can be sufficiently described using perhaps the axial quadrupole moment and maybe one other multipole moment which depends on the specific system, often the axial octupole moment or triaxial quadrupole moment. Of course, it is well-understood that some physics may be obscured in a limited collective space (see \cite{Dubray2012}). Thus, one's choice of collective coordinates is dependent on what physics are deemed important or relevant, and which aspects can be safely neglected.

However, although various attempts have been made to demonstrate the validity of this assumption, our work represents the first published instance of a 4D potential energy surface calculated self-consistently. Furthermore, given the recent demonstration of the importance of pairing correlations as a collective ``coordinate'' of the system, ours will feature pairing as part of the collective space, and its impact compared to other collective coordinates will be evaluated.

We used 30 harmonic oscillator shells and 1500 states

\subsection{Cluster Decay}

Experimental instances of super-asymmetric fission:
M. G. Itkis 1985, Z Phys A 320 - no assessment of the cause of highly-asymmetric fission, but likely related to 132Sn (there nuclei would tend to fission symmetrically, but with a slight bump around mass A=140-145)
D. Rochmann Nucl Phys A 735 (2004) - driven by shell structure of lighter fragments
I M Itkis, J Phys Conf Ser 515 (2014) 012008 - cluster radiation by another name

AKA “Lead Radioactivity” sometimes in the literature
To predict cluster half-lives, some people take it as a very heavy alpha emission, and others a very asymmetric fission
Warda looks at the N/Z ratio of known cluster emitters (or really of lead-208), and then extrapolates it out to SHEs. That’s how he decided which superheavies to compute
PRC 86 (2012) 014322
Nucl Phys A 944 (2015) 442 (with Baran and others)


\subsection{Synthesis of Og}

They found 3 (and possibly 4) instances in the original Dubna run. Then there was a secondary run at Oak Ridge that was about the same: something like 3 alpha events and a possible fission event. (Another Og paper is being prepared (Nathan Brewer, et al), which has a similar decay chain but a shorter half-life ($\sim$0.185 ms); Detected a 10.6 MeV recoil event, followed 78 microseconds later by a second decay event in the same pixel ($\sim$140 MeV), which is a candidate for SF)

\subsection{Competition with Alpha Decay}

%Alex Brown has something: PRC 46, 2, 811-814 (1992) https://link.aps.org/doi/10.1103/PhysRevC.46.811
%Ion’s is in Rom Journ. Phys. 62, 303 (2017) http://www.nipne.ro/rjp/2017_62_7-8/RomJPhys.62.303.pdf
%Roderick Clark: https://journals.aps.org/prc/abstract/10.1103/PhysRevC.97.024333
%Chinese review paper on alpha decay models: https://journals.aps.org/prc/abstract/10.1103/PhysRevC.92.064301

Recent efforts to synthesize superheavy elements (SHE) have successfully produced the isotope $^{294}$Og, which has been confirmed via its alpha-decay chain. In both experiments, the researchers found evidence of alpha decay, but both also noted the possible observation of decay via spontaneous fission. This suggests the possibility that $^{294}$Og might have a similar decay time with respect to both alpha-decay and spontaneous fission.

There has been an expectation (for some reason?) that cluster emission (known also in the literature as cluster radioactivity, lead radioactivity, cluster decay, heavy-particle radioactivity, ???) might play an important role in the fission of superheavy elements, suggesting that even for such large nuclei (where the Coulomb repulsion is strong), shell structure of the prefragments still drives the determination of the fragments.

\cite{Poenaru2011, Poenaru2012} - In this paper they propose changing/extending the concept of Heavy Particle Radioactivity or Cluster Radioactivity. Also they apply some model to HPR/CR in SHE.

``A larger number of observed spontaneous fission ac- tivities enabled the establishment of a global dependency of spontaneous fission half-lives (TSF ) and the fissility of a nucleus, expressed by the ratio Z2/A which had been realized already by Seaborg [111] and also by Whitehouse and Galbraith [112]. The data, available at that time indicated for even-even nuclei an exponential dependence of the fission half-lives from Z2/A. From an extrapolation of the trend it was concluded, that a nucleus will become instantaneously unstable against nuclear fission at Z2/A ≈ 47, which was set in correspondence with a half-life of ≈ 10−20 s. Interestingly, the heaviest nucleus reported to be synthesized so far, 294118 (294Og) [65], has a value Z2/A ≈ 47.36. The half-life is given as T1/2 =0.69+0.64−0.22 s. Up to now four α decays, but no spontaneous fission was observed [65].'' - from \cite{Heßberger2017} - Og is anomolous in that it violates this extrapolated trend (as would, I am sure, most SHE).

Whether or not this PES is able to reasonably describe the CN experiments which so far have produced $^{294}$Og is uncertain, because such large compound nucleus expectation energies as appear in experiment may have quite a large effect on the topology of the PES \cite{Pei2009}

On the theory side, there have been several attempts to compute spontaneous fission half-lives and alpha-decay half-lives for many superheavy nuclei, and in many cases it is predicted that the two lifetimes will be comparable \cite{Poenaru2011, Poenaru2012, Zhang2018} [Zhang was an application of several universal CR and alpha decay models to the SHE, in order to see if the predictions, too, were universal]. These previous works have tended to rely on phenomenological models which have been tuned to smaller, more stable nuclei. Thus, it is difficult or impossible to assess these models' predictive power in the region of SHE. Thus, a goal of this work is to bring the full predictive framework of self-consistent nuclear density functional theory to bear on the problem of spontaneous fission in the SHE $^{294}$Og. This approach is relatively young in the world of nuclear fission models, but it is already producing quality results for a variety of nuclei in different regions of the nuclear chart (see, for instance, \cite{Mcdonnell2014, Sadhukhan2017, Sadhukhan2016, Tsekhanovich2018}). Some attempts in the region of SHE have already been made, using Skyrme and Gogny functionals in a 2D space \cite{Reinhard2017, Giuliani2017, Warda2012, Baran2015}.

Within these models, spontaneous fission lifetimes tend to be considerably larger than alpha decay lifetimes, ranging from $\frac{\tau_{SF}}{\tau_{\alpha}}\approx10^{-10}$ in \cite{Baran2015} and \cite{Reinhard2017} to $\frac{\tau_{SF}}{\tau_{\alpha}}\approx10^{-20}$ in \cite{Warda2012}. However, it was shown in \cite{Sadhukhan2014} that pairing correlations treated as a dynamical variable can have a substantial impact on spontaneous fission lifetimes. That is explored in the case of $^{294}$Og here.

\section{Method}

Our calculations were performed within the framework of nuclear density functional theory using Skyrme and Gogny energy density functionals. In the Skyrme case, the parameterization UNEDF1-HFB \cite{Schunck2015} was used, and pairing correlations were described using a density dependent pairing interaction. To assure convergence despite the high density of states, the DFT solver HFODD was used with 30 harmonic oscillator shells and 1500 states allowed in the calculation. Calculations were performed in a 4D collective space consisting of 3 shape coordinates, $(q_{20}, q_{30}, q_{22})$, and, given the importance of dynamic pairing fluctuations demonstrated in \cite{Sadhukhan2014}, $\lambda_2$. To demonstrate model independence, another set of calculations was performed using the Gogny energy density functional D1M in the two-dimensional collective space described by coordinates $(q_{20},q_{30})$.

It is seen in many models that introducing triaxiality as a degree of freedom can often be energetically-favorable, sometimes lowering saddle points by as much as 3 MeV; however, dynamic calculations in which the collective inertia is considered together with the potential energy surface have found that dynamical pathways usually tend to tunnel through barriers rather than break axial symmetry. This competition was explored for SHE in \cite{Gherghescu1999}, with the conclusion that triaxiality plays a fairly insignificant role in determining the half-life of elements below $Z=120$. However, another recent paper (https://arxiv.org/abs/1803.04616v2) suggests that triaxiality might significantly lower the second barrier. Regardless, we included $q_{22}$ in our calculations. It may also be the case that isotopes which are oblate-deformed in their ground state may pass through triaxial configurations on their way to greater elongations.

The basis of the model is the assumption that spontaneous fission can be treated such that the lifetime is proportional to $e^{-P}$, where $P$ is the transmission probability through some barrier.

The collective inertia of the system was computed using the nonperturbative ATDHFB cranking approximation in the Skyrme case, and perturbative ATDHFB with cranking and perturbative GCM with cranking in the Gogny case \cite{Baran2011}. The tunneling is described using the WKB approximation, in which the tunneling path $L(s)$ was computed by using the dynamic programming method to minimize the collective action

\begin{equation}
S(L) = \int_{s_{in}}^{s_{out}} \frac{1}{\hbar}\sqrt{2\mathcal{M}_{eff}\left(V_{eff}(s)-E_0\right)}ds
\end{equation}

\noindent where $\mathcal{M}_{eff}$ is the effective inertia and $V_{eff}$ the effective potential energy along $L(s)$. Following the formalism of \cite{Sadhukhan2013}, the half-life is computed via $T_{\frac{1}{2}} = \ln 2/nP$, where $n=10^{20.38}s^{-1}$ is the number of assaults on the fission barrier per unit time and the penetration probability $P$ is given by

\begin{equation}
P = (1 + exp[2S(L)])^{-1}
\end{equation}

\noindent Finally, after computing the action at many points along the outer turning line, the final fragment yields were determined by evolving the system many times via Langevin dynamics, following the work done in \cite{Sadhukhan2016}.

\section{Langevin dynamics}


\section{Fragments and the Nucleon Localization Function}
An improved scission criterion would go beyond simply counting the number of particles in the neck. To help with this, we have a tool at our disposal which helps us to understand correlations that affect fission dynamics. This is called the nucleon localization function, and it allows us to visualize the prefragment nuclear shell structure which largely determines the identity of fission fragments \cite{Zhang2016}.

The nucleon localization function shows that some prefragments can be very well-formed even when the neck is large, while in another case the neck might be small but the prefragments, poorly-defined \cite{Sadhukhan2017}. A better scission criterion should take into account, or at least be compatible with, the insights gained from the nucleon localization function. As noted in \cite{Younes2009}, fragment properties on either side of the scission line may differ drastically. This is because shell structure is not well-described geometrically. Our localization measure offers an alternative scheme for identifying fragments before the scission line (see \cite{Sadhukhan2017}). Since it is based on the underlying quantum shells, it is less sensitive to fluctuations and particle rearrangements late in the evolution.
 % Two fission modes in 178Pt
\chapter{R-process}

\maketitle
I'll say some stuff about r-process nuclei here

Fission inputs to r-process network calculations % Cluster decay in 294Og
\chapter{Outlook}\label{chap:Outlook}
\maketitle

In this chapter, it would be great to talk to everyone you know (Witek, Samuel, Jhilam, Nicolas, Michal, and so on) to get a better feel for what kinds of issues need to be addressed next. You've already got sort of a rudimentary understanding (see your Google Keep note for starters), but it might be good to get some outsider perspective. This will be especially important as you start looking for postdocs, and \textit{especially} especially if you end up looking for postdocs in nuclear theory, but not necessarily nuclear fission.

As I said in chapter \ref{chap:Intro}, ``Finally, in chapter \ref{chap:Outlook} we discuss the current state of the field, and, based on our experience, offer insights for guiding future developments in the field.''

We definitely need a better handle on the inertia. The perturbative inertia is easy to compute, but not terribly reliable. The non-perturbative inertia can certainly do better, but as it is computed now (using finite differences) it is subject to numerical artifacts and instabilities (dependent on the level of convergence of the individual densities, the coefficient mutlipliers, different basis sizes) and actual physics, such as level crossings which manifest in projections from a higher-dimensional space.

UNEDF1 seems to underestimate fission barrier heights (artificial though the concept may be; the main impact is probably that lifetimes are underestimated). It also turns out to be a headache to work with, making convergence quite a challenge sometimes (any cases in particular, like for highly-deformed or heavy or octupole-deformed nuclei or something?). Better functionals might hope to better capture the physics, and one can hope they are easier to work with. % R-process
\chapter{Conclusions}\label{chap:Outlook}
%In this chapter, it would be great to talk to everyone you know (Witek, Samuel, Jhilam, Nicolas, Michal, and so on) to get a better feel for what kinds of issues need to be addressed next. You've already got sort of a rudimentary understanding (see your Google Keep note for starters), but it might be good to get some outsider perspective. This will be especially important as you start looking for postdocs, and \textit{especially} especially if you end up looking for postdocs in nuclear theory that aren't specifically nuclear fission.

The overarching goal of the project, in which this dissertation is a part, is to describe spontaneous fission yields in a self-consistent framework based on microscopic nuclear physics. We highlight spontaneous fission in three distinct and relatively-unexplored regions of the nuclear chart. New phenomena such as the observed asymmetry of fragments in the neutron-deficient sub-lead region, superheavy element synthesis and decay, and r-process nucleosynthesis force us to grapple with our understanding of fission and ultimately lead to greater insight and understanding of the fission process. New computational tools are supporting these developments by allowing us to use more efficient, more detailed models in our calculations.

In Chapter~\ref{chap:178Pt}, we estimated the peak of the fragment distribution corresponding to spontaneous fission of {\Pt}. In addition, we were able to qualitatively characterize properties of the fragments using properties of the PES, in a way that was compatible with experimental data.

In Chapter~\ref{chap:294Og}, we used Langevin dynamics to estimate the fission yields belonging to the superheavy element {\Og}. We predicted that this nucleus may fission according to an exotic form of fission called cluster emission, which is driven by the strong shell closures associated with $N=126$ and $Z=82$. We also took this calculation as an opportunity to test the robustness of microscopic fission calculations using Langevin dynamics, and we found that the peak was robustly predicted to within a few nucleons even as the inputs changed.

In Chapter~\ref{chap:rprocess}, we began using a new method for estimating fragment yields (see Appendix~\ref{append:Fragments}) to calculate fission fragment distributions that are relevant to the astrophysical r process. This method is designed to be efficient, which will allow it to be used for survey-level calculations that might later be used as inputs in r-process network calculations.

With that said, there is much still to accomplish on the way to a comprehensive microscopic description of spontaneous fission. This includes additional physics modeling as well as basic technical developments. The following few pages give an overview of some of these future prospects.

\section{Improved model fidelity}
Qualitatively and quantitatively, microscopic models are useful for understanding the complex physics of the fission process. To be useful for many applications, however, the level of quantitative agreement with experiment needs to improve significantly. Using the case of $^{240}$Pu as a benchmark~\cite{Sadhukhan2016}, the Langevin approach can predict spontaneous fission fragment yields to within around 30\% accuracy. Depending on the application, this number must be brought down to ${\sim}5\%$ to be useful.

Half-lives, which are an essential defining element of spontaneous fission, are especially difficult to estimate. Assuming that the minimum action is computed to the same 30\% accuracy as the yields (which is admittedly a rough estimate, since these quantities depend on two different regions of the PES with drastically different dynamics), then, because of the exponential dependence of the half-life on the action, the half-life may be off by several orders of magnitude.

%... Improving precision (and scission!) and quantifying uncertainties will also prepare the way for other observables, such as spin and kinetic energies.

% There are maybe better ways to compute $E_0$? \verb|\cite{???}| But I don't know what it is, so I'll just not mention it

Since fission lifetimes are largely determined by the topography of the PES through which the nucleus must tunnel, it is important to improve predictions of both the potential energy and the collective inertia. The perturbative inertia is easy and fast to compute, but not expected to be as reliable as the non-perturbative inertia. And as discussed in Section~\ref{sect:M_numerical}, the finite difference formulae from which the non-perturbative inertia is often computed are subject to errors due to level crossings and numerical artifacts. One tool that would eliminate the need for finite-difference derivatives is automatic differentiation. Automatic differentiation essentially keeps track of arithmetic operations as they are performed and combines their derivatives according to the chain rule. Because this is performed as the code runs, it adds almost no additional runtime and no numerical error. Once implemented, automatic differentiation would allow derivatives to be computed with machine precision, without the need to compute any additional nearby points. The chief drawback is that automatic differentiation can be difficult to implement.

As for the potential energy, the current state-of-the-art functional used for fission - UNEDF1 - tends to underestimate the height of saddle points on the PES, which in turn has the effect of artificially reducing half-lives. One might suggest refitting the functional using additional fission data, but unfortunately, the UNEDF2 project~\cite{Kortelainen2014} concluded essentially that Skyrme-like functionals have been pushed to their limits. A next-generation set of functionals based on the UNEDF project has been proposed which uses an expansion of the density matrix and a link to modern effective forces, both of which can be systematically improved~\cite{NavarroPerez2018}. This set of functionals has not been extensively tested, save for some initial benchmarking, so it is not clear whether there will be any added benefit to using them. It may be that new functionals such as these will better reproduce fission data, but it should be noted that, while the calculations presented here would not be affected, introducing an explicit density-dependence can cause problems for beyond-mean field corrections~\cite{duguet2009, dobaczewski2007, anguiano2001}. %It also turns out to be a headache to work with, making convergence quite a challenge sometimes (any cases in particular, like for highly-deformed or heavy or octupole-deformed nuclei or something?).

It was assumed throughout this dissertation that fissioning nuclei behave as superfluids maintained at temperature $T=0$. However, a proper description of fission from an excited compound nucleus as in the case of Chapter~\ref{chap:178Pt}, or neutron-induced fission as in Chapter~\ref{chap:rprocess}, requires a finite-temperature formalism. This is discussed in Appendix~\ref{append:TD-ATDHFB}.

A known problem when constructing PESs is the existence of discontinuities, which occur when one reduces a complex system with hundreds of degrees of freedom into only a few collective coordinates~\cite{dubray2012}. This can obscure important physics, such as saddle point heights and the complex physics of scission. Brute force computation with a larger number of collective coordinates (such as we did in Chapter~\ref{chap:294Og}) will become more accessible as computers continue to become more powerful. However, we might see a faster turnaround on investment if we start using collective coordinates that better describe the shape of fissioning nuclei. For instance, a set of coordinates $D, \xi$ was proposed as an alternative to the multipole moments $Q_{20}$ and $Q_{30}$ in~\cite{younes2012}, where $D$ is the distance between prefragment centers of mass and $\xi = (A_R-A_L)/A$ characterizes the fragment asymmetry. Using these coordinates, the authors found that they were better able to describe scission configurations than if they had used multipole moments instead. % Would non-geometric collective coordinates be better-suited for fission? Like some kind of interaction energy that is unique to that process?

% To that end, it is worth mentioning as well Fragment identification (our localization paper, Marc Verriere's method; you might also mention that this is not an issue in TDDFT, but there you've only got one single fragment pair) - I don't think Marc's work is published yet, though, and I don't have anything better to suggest right now

Finally, to make the calculations described in this dissertation fully self-consistent will require the development of a self-consistent description of dissipation. In addition to contributing to a more reliable model, it may be that a better understanding of the exchange mechanisms between intrinsic and collective degrees of freedom could allow one to more reliably model fragment kinetic and excitation energies. The topic of dissipative motion in quantum systems is being studied theoretically~\cite{Koch2008, Lacroix2008, Hupin2010, Sargsyan2010, Bulgac2018a}, but the ideas are not yet sufficiently well-developed for quantitative fission calculations.

% A proper understanding of the friction tensor (as it were) may also help with the calculation of neutron emissions, according to this: ``The  influence  of  friction  on  the  fission  rate  was studied in papers [57, 58]. It was shown there that dissipative  effects  might  lead  to  emission  of  more  neutrons from fissioning nuclei than what is predicted by equilibrium  statistical  models.'' (\verb|https://link.springer.com/article/10.1134%2FS1063779610020012|).

% Some other papers you can look at for this might be \verb|https://doi.org/10.1142/S0218301398000105|, \verb|https://doi.org/10.1016/j.physletb.2007.09.072|, and \verb|https://doi.org/10.1016/0375-9474(79)90559-1|.

% I don't think anyone is really looking to do this, beyond the TXE+TKE info which they can then plug into their Hauser-Feshbach codes
%\section{Computing other fission observables}
%It is tempting and exciting to push a model to its limits, and to see how generally it can be used. In this case, that means figuring out how to predict other observables in a self-consistent framework beyond half-lives and fragment yields. These might include fragment energetics (kinetic and excitation energies), fragment angular distributions, spins, prompt neutron multiplicities, prompt neutron and gamma energy spectra. It may soon be feasible to accurately predict fragment energies within the adiabatic approximation~\cite{Younes2011}; for everything else, the issue of how to compute these observables in a self-consistent framework is still an open question. In fact, many of these (especially particle emissions) may turn out to be forever beyond the scope of self-consistent models, delving deep into the realm of statistical physics. In that case, it would be interesting to see how these statistical models perform with self-consistent inputs.

% Also, temp-dependence...


\section{Computational tools}
As we look for and find new ways to study and apply fission, we might consider the availability of new computational tools, such as machine learning and GPUs.

As stated earlier, our biggest bottleneck is the calculation of the PES. The expense of calculations has the side effect of forcing us to use small collective spaces with relatively-few collective variables, which may mean a loss of important physics. Therefore, it is worth investing in methods to reduce the burden of PES calculations. One potential solution is a machine learning method called speculative sampling. Speculative sampling works well for problems that have an expensive computational model, and calculations that must be performed for many inputs or in a large parameter space. The essence of this technique is to create an emulator for your model using machine learning, then initiate many calculations using the expensive model and terminate early any calculations which do not appear to diverge strongly from the emulator. In this way, we don't waste time calculating things that the emulator can already predict, and instead focus our efforts on those portions of the PES which provide new information. After this set of calculations converges, the emulator can be retrained using the new data and the procedure is repeated.

This could be implemented for our problem in the following way: Start by using Monte Carlo sampling to begin constructing a PES in a large collective space using a traditional DFT solver, and then use those results to train a PES emulator. Then use Monte Carlo sampling to start a new round of PES calculations, only this time compare the results from the DFT solver in real-time to the emulator's prediction. As long as the DFT solver diverges from the prediction it should be allowed to continue, but once it appears that the DFT calculation agrees with the prediction the calculation terminates. The emulator should then be retrained using the new data, and the feedback cycle continued until the PES is predicted sufficiently-well by the emulator. Although traditional simulations will likely continue to be performed on traditional CPUs, machine learning performs well on GPUs. That means this technique will be able to take advantage of many of the newest architectures.

For microscopic calculations to become practical for r-process calculations, though, would require an efficient way of estimating fission properties on a large scale, ideally covering the whole nuclear chart. To do this using conventional calculations will still be a tremendous undertaking, even with the potential speedup from speculative sampling. Another machine learning paradigm called ``transfer learning'' might be useful here. In transfer learning, one starts with a set of data which is large, but imperfect. For fission, this might mean yields computed globally via a phenomenological model such as GEF~\cite{Schmidt2016}. This data is used to train a model, such as an artificial neural network, so that essentially one has an emulator that understands the physics of the cheap model. This knowledge is then \textit{transferred} (hence the name) to a new model with new data, which in our case might be experimental data or even the results from detailed microscopic calculations. In a neural network with many layers, this transfer might take place by retraining the model while holding all but the last two or three of layers constant. In our example, the first several layers contain most of the basic physics of fission, and then the last few layers contain corrections and improvements to the simple model.

% See here for where you got the ideas: https://bout.llnl.gov/content/assets/docs/workshops/2018/2018-08-17_Spears.pdf

\section{Outlook}
It is an exciting time to study fission. New phenomena are being discovered as fission experiments are performed in unfamiliar regions of the nuclear landscape, and there is a healthy feedback loop between theory and experiment thanks in great part to modern computational tools. It is, in many ways, the culmination of years of theoretical development, but there is still much to be done. % Outlook

%%%%%% LANDSCAPE PAGES  %%%%%%
%% To produce graphics or tables in landscape mode,
%% begin by removing the "%" in the next two lines.
%\begin{landscape}
%\thispagestyle{empty}
%% The contents of the page can be centered using the center environment
%% or the \centering command. Insert either a table with the tabular environment
%% or input a graphics file.
%% Use \captionof{table}{caption_text} (or figure in place of table) 
%% to create the caption for a short table or a figure. 
%% Insert a long table in a table environment. It disables double spacing
%% which is permitted for long tables. Finally remove the "%" from the next line.
%\end{landacape}

%%%%%%%    APPENDICES    %%%%%%%%%%
%% If you wish to include just one appendix, remove the "%" 
%% in front of \appendix below. To include two or more appendices,
%% remove the "%" in front of \appendices.
%\appendix
\appendices
\chapter{Temperature-Dependent ATDHFB Collective Inertia}\label{append:TD-ATDHFB}


Everything which was shown in this dissertation assumed that the system was maintained at temperature $T=0$ and the nucleus behaved as a superfluid below the Fermi surface. However, in many environments (such as a neutron star merger or a nuclear blast) there may be quite a bit of excitation energy imparted to the system, which would raise the temperature above the Fermi surface. In this case, pairs may be broken and the topology of the potential energy surface may change (see, for instance, \cite{Mcdonnell2014}). In this case, the collective inertia of the system is changed, too, as shown below. % Appendix A: Fragment ID
\chapter{Temperature-Dependent ATDHFB Collective Inertia}\label{append:TD-ATDHFB}


Everything which was shown in this dissertation assumed that the system was maintained at temperature $T=0$ and the nucleus behaved as a superfluid below the Fermi surface. However, in many environments (such as a neutron star merger or a nuclear blast) there may be quite a bit of excitation energy imparted to the system, which would raise the temperature above the Fermi surface. In this case, pairs may be broken and the topology of the potential energy surface may change (see, for instance, \cite{Mcdonnell2014}). In this case, the collective inertia of the system is changed, too, as shown below.

Be sure to discuss the complications which arise in the finite temperature formalism, as promised in Chapter \ref{chap:Numerical}. In essence, you end up dividing by terms which are very small. You can avoid dividing by zero by introducing a cutoff. If the cutoff is too large, you lose some of the data in the tail. If the cutoff is too small, you divide by numbers that are smaller than the noise in the density. There are actual numbers in your dudeman4 Google Drive, in a file called Inertia Tensor Convergence. % Appendix B: Temperature-dependence of collective inertia
\chapter{List of my contributions}\label{append:Contributions}

\begin{itemize}
	\item \textbf{Observation of the competing fission modes in $^{178}$Pt}
	\begin{itemize}
		\item \textit{Phys. Lett. B 790, 583-588 (2019)}
		\item Performed PES and localization calculations
		\item Created graphics for Figure 3
		\item Suggested several revisions to the text of the article
	\end{itemize}
	\item \textbf{Colloquium: Superheavy elements: Oganesson and beyond}
	\begin{itemize}
		\item \textit{Rev. Mod. Phys. 91, 011001 (2019)}
		\item Generated Figure 12
		\item Helped write Section VI: Fission
		\item Suggested comments and revisions to the text
	\end{itemize}
	\item \textbf{Cluster radioactivity of $^{294}$Og}
	\begin{itemize}
		\item \textit{Phys. Rev. C 99, 041304 (2019)}
		\item Lead author on paper/wrote first draft
		\item Did all UNEDF1$_{\mathrm{HFB}}$ calculations and SkM* Langevin calculations
		\item Generated all figures
	\end{itemize}
	\item \textbf{Microscopic Calculation of Fission Mass Distributions at Increasing Excitation Energies}
	\begin{itemize}
		\item \textit{Forthcoming CNR2018 Proceedings...}
		\item Helped derive [one of the equations]
		\item Contributed to discussion of ???
	\end{itemize}
	\item \textbf{Fission Tools}
	\begin{itemize}
		\item \textit{Suite of codes which extend the functionality of HFODD, HFBTHO, and other DFT solvers to the problem of nuclear fission.}
		\item \verb|https://gitlab.com/zachmath/fission_tools|
		\item Maintainer, 62 commits
		\item Converted old codes from Fortran77 to Fortran90
		\item Implemented shared memory (OpenMP) and distributed memory (MPI) parallelism
		\item Improved documentation, flexibility, and user-friendliness
		\item Created several Python-based utilities for plotting and file handling
		\item Increased functionality, such as by increasing from 2D to 3D or 4D
	\end{itemize}
	\item \textbf{PES Tools}
	\begin{itemize}
		\item \textit{Python framework for handling potential energy surfaces as XML files}
		\item \verb|https://gitlab.com/schuncknf/pes_tools|
		\item Developer, 25 commits
		\item Added a \verb|point| class with various methods to use on a single point of a PES
		\item Interface to \verb|fission_tools|
		\item Various updates bugfixes
	\end{itemize}
	\item \textbf{DFTNESS}
	\begin{itemize}
		\item \textit{DFTNESS (Density Functional Theory for Nuclei at Extreme ScaleS) is a computational framework to solve the equation of nuclear density functional theory. It is based on the two solvers HFBTHO and HFODD.}
		\item \verb|https://gitlab.com/schuncknf/dftness|
		\item \href{https://gitlab.com/schuncknf/dftness}{https://gitlab.com/schuncknf/dftness}
		\item Developer, 60 commits
		\item OpenMP parallelization of subroutine \verb|QMULCM|
		\item Various updates and bugfixes
		\item NOTE: Most of these commits were related to \verb|fission_tools|, before that was all moved to a separate repository
	\end{itemize}
\end{itemize} % Appendix C: List of my contributions

%% In either case to start your first appendix, which will be labeled
%% as Appendix A, just type \chapter{<appendix 1 name>}
%% and enter the text of the appendix as you would a chapter,
%% with one exception. If you use any subdivisions, such as 
%% section, subsection etc., use the starred version; that is,
%% \section*{section name}.  Such subdivisions are not to be 
%% listed in the Table of Contents.

%%%%%%% A NOTE ABOUT APPENDICES %%%%%%%%%
%% Some appendices may be single-spaced such as survey examples
%% or letters. See the Graduate School's formatting guide for details.
%% To single space an appendix first remove the "%" from 
%% the following two lines.
%\end{doublespace}
%\chapter{<appendix name>}
%% Insert the name of the appendix.
%% Insert the text of the appendix.
%% Remove the "%" from the following line.
%\begin{doublespace}
%% Any text entered now will be double spaced.

\end{doublespace}

%%%%%%  Bibliography %%%%%
%% A bibliography is required. By default it is called, "Bibliography"
%% You may use �LITERATURE CITED�, �WORKS CITED� or �REFERENCES� 
%% instead of �BIBLIOGRAPHY� if that is the convention in your discipline. 
%% To do so, copy and paste your choice into the empty argument 
%% of the following command and remove the "%".
%\renewcommand{\bibname}{}

%% The bibliography may be made using BibTeX.
%% To do so the necessary commands must be entered in the 
%% preamble and here.
\bibliography{main}
\bibliographystyle{apsrev4-1} % I don't know what bibliography style to use, so this is a placeholder

%% If the Bibliography is made from scratch,
%% remove the "%" in front of \begin{thebibliography}{???}
%% replacing the ??? with the appropriate entry and 
%% remove the "%" in front of \end{thebibliography} below.
% \begin{thebibliography}{???}
%%  Enter the bibliography here.
% \end{thebibliography}
%% In either case, the bibliography is automatically entered
%% in the Table of Contents.
\end{document}

%%%%%% FINAL COMMENTS %%%%
%% Before submitting your dissertation to the Graduate School
%% Make sure there isn't any text in the right margins. To do 
%% in the .log file look for error messages beginning with, 
%% "Overfull \hbox ". 

%% Once your document has been filed with the Graduate School,
%% if you wish to produce a single spaced version of your document, 
%% find and remove the commands \begin{doublespace}
%% and \end{doublespace} above.
