\chapter{Two fission modes in $^{178}$Pt}\label{chap:178Pt}

\section{Asymmetric fission in the region of $^{180}$Hg}
As mentioned in the introduction, fission has been studied most carefully in the region of the actinides (Z=90 to Z=103), as many naturally-occurring isotopes in this region are fissile. Within this region, there is a characteristic tendency for fission fragment yields to be asymmetric (that is, one light fragment and one heavy fragment), with the heavy peak centered around $A\approx140$. This has been understood as a manifestation of nuclear shell structure in the prefragments: doubly-magic $^{132}$Sn drives the nucleus towards scission, and once the neck nucleons are divided up between the two fragments, the heavy fragment distribution peaks near A=140. As one moves to the lower-Z actinides, however, this tendency becomes less and less pronounced as yields tend to become more symmetric. Until recently, it was generally believed that below thorium, yields would continue to be symmetric. For sub-thorium isotopes, there was no doubly-magic nucleus candidate that could drive the system toward asymmetry as there is with actinides.

However, it was reported in a 2010 study \cite{Andreyev2010} that neutron-deficient $^{180}$Tl undergoes beta-delayed fission, leading to intermediate state {\Hg} which then decays into two fragments of unequal mass. This finding triggered a flurry of theoretical papers hoping to describe this new and unexpected phenomenon (for instance, see \cite{Warda2012,Moller2012,Mcdonnell2014,Ichikawa2019}). A follow-up study using $^{178}$Tl \cite{Liberati2013} further established this as a region of asymmetric fission, and not just a one-time occurrence. Since then, other nuclei in the region have been studied using a variety of reactions and techniques, and the finding is the same. 

%Nuclei in this region have a number of unique features which make them interesting for study, even aside from the unexpected fragment asymmetry. Predicted fission barrier heights in this region are relatively-low (of the order of ~12 MeV), making them suitable for study using low-energy techniques such as $\beta$-delayed fission (maybe \cite{Andreyev2013} and the work at ISOLDE at CERN?) or Coulex-induced fission (maybe \cite{Martin2015} and the SOFIA (Studies On FIssion with Aladin) experiment/project/campaign). On the other hand, it has been found that compound nuclei formed in this region from particle-induced reactions tend to have high excitation energies, even for beam energies near the Coulomb barrier. This combination makes the region particularly well-suited for studies involving a variety of excitation energies.

Later experiments have shown that, unlike the case of actinides where shell structure and fragment asymmetry are ``washed out'' at high excitation energies, mass asymmetric fragment distributions are a persistent feature of this mass region for various excitation energies (see \cite{Andreyev2018} and references therein). An overview of nuclei in the region of {\Hg} which have been experimentally studied (as of 2016), including the experimental technique used, is shown in Figure \ref{fig:178ptregion}.

\begin{figure}
	\centering
	\includegraphics[width=0.9\linewidth]{./TeX_files/178Pt_region}
	\caption[Survey of fragment yields near $^{180}$Hg]{Fragment yields for several nuclei ranging from actinides, where primary fission yields tend to be asymmetric, down to near-thorium, where yields become more symmetric, and finally to the region near neutron deficient {\Hg}, where asymmetry returns. Figure from \cite{Andreyev2018}.}
	\label{fig:178ptregion}
\end{figure}



\section{Multimode fission of $^{178}$Pt}

One particular follow-up experiment was performed to investigate the spontaneous fission of {\Pt} \cite{Tsekhanovich2019}, which differs from {\Hg} by 2 protons. This system was studied at various excitation energies and found to fission consistently in a bimodal pattern, as shown in Figure \ref{fig:178ptexptdata}. Of the sample measured, roughly 1/3 of cases were found to fission symmetrically, while the other 2/3 fissioned asymmetrically with a light-to-heavy mass ratio of approximately 79/99. Furthermore, it was observed that mass-asymmetric fragments tended to have higher kinetic energies than symmetric fragments.

\begin{figure}
	\centering
	\includegraphics[width=0.95\linewidth]{TeX_files/178Pt_expt_data}
	\caption[$^{178}$Pt experimental data]{After projecting the fission fragment mass vs total kinetic energy (TKE) correlation (a) onto the TKE axis, the TKE distribution is deconvoluted into high- and low-energy components, centered around the values TKE$^\mathrm{high}$ and TKE$^\mathrm{low}$ (b). The corresponding mass distributions are fitted to a double (c) and single (d) Gaussian, respectively. This procedure is repeated for three different compound nucleus excitation energies E* (e-g). Figure from \cite{Tsekhanovich2019}}
	\label{fig:178ptexptdata}
\end{figure}

To better interpret the results of this experiment, DFT calculations were performed using the functionals {\hfb} \cite{Schunck2015} and D1S \cite{Berger1989}. These calculations involved computing a PES using the collective coordinates $Q_{20}$ and $Q_{30}$. The {\hfb} PES is shown in Figure \ref{fig:178ptunedf1pes}, while the D1S PES is in Figure \ref{fig:178ptd1spes}. A calculation with full Langevin dynamics was not performed; however, the static (minimum-energy) pathways shown in the figures correspond to a fragment split $A_L/A_H \approx 80/98$.

\begin{figure}
	\centering
	\includegraphics[width=0.7\linewidth]{TeX_files/178Pt_unedf1_pes.jpg}
	\caption[UNEDF1-HFB potential energy surface for $^{178}$Pt]{UNEDF1-HFB potential energy surface for $^{178}$Pt. Note the two different trajectories ABCD and ABcd and their corresponding localizations.}
	\label{fig:178ptunedf1pes}
\end{figure}

Also shown in Figure \ref{fig:178ptunedf1pes} are nucleon localization functions (recall Section \ref{sect:locali}) corresponding to various marked configurations in the PES. Along the symmetric path (ABcd in the figure), the fragments appear highly-elongated, even shortly before scission. Since elongation tends to minimize the Coulomb repulsion between fragments, then this configuration might be expected to lead to fragments with relatively low kinetic energies. On the other hand, compact fragments such as those in ABCD will tend to have a larger Coulomb repulsion. Fragments will be propelled away from one another with greater force, resulting in fragments with a higher kinetic energy, which is in qualitative agreement with experiment.

Comparing the {\hfb} PES in Figure \ref{fig:178ptunedf1pes} to the D1S PES in Figure \ref{fig:178ptd1spes}a, one may note that, despite the inherent differences between the functionals, the overall topology of the PES is similar in both cases. In fact, the topology in both cases is quite flat, which suggests (in agreement with the observed data) the possibility of a competition between symmetric and asymmetric fission modes. Additionally, as shown in Figure \ref{fig:178ptd1spes}b, there appears an additional channel corresponding to compact symmetric fragments. This channel is higher in energy than the elongated symmetric pathway and is blocked by a $\sim4$ MeV saddle point, but it is conceivable that this channel might be more easily accessed by other isotopes in this region.

\begin{figure}
	\centering
	\includegraphics[width=0.7\linewidth]{TeX_files/178Pt_D1S_pes.jpg}
	\caption[D1S potential energy surface for $^{178}$Pt]{D1S potential energy surface for $^{178}$Pt. Note also the additional information about the hexadecapole moment.}
	\label{fig:178ptd1spes}
\end{figure}



\section{The physical origin of fragment asymmetry in the region of $^{180}$Hg}

Why is there a region of symmetric fission below thorium?

(These are notes from the 178Pt paper draft. Not mine, of course, but they have some good points to address): ``Namely, the PES are predicetd to be flat and much less structureless, and defined predominantly by the large liquid drop/macroscopic contribution, rather than by relatively small microscopic effects. Due to this, FFMDs exhibit fairly low dependence..  [refer to 180Hg PLB, as one example].

``(this was an answer by Witek, when somebody asked a question to my talk at Tsukuba - why the lead region is less sensitive to temperature.. the answer was - there is no 'barrier' in a sence, it's just flat/thick macroscopic surface, hardly influenced by shell effects.. so, even if one heats it up, tiny shell effects will be gone, but the main underlying macroscopic part will remain).''

Peter Moller argues in the concluding discussion of (https://link.aps.org/doi/10.1103/PhysRevC.85.024306 \cite{Moller2012}) that we can't really use the fragment/prefragment shell structure arguments in this region, and thus that we have yet to identify all the essential physics which determines fragments. He says the yields are given (at least in this case) by subtle interplays in local regions of the potential energy surface. He also has another paper \cite{Ichikawa2019}.

Witek, Michal Warda, and Staszczak argue in Section IV. \textit{Prescission Configurations} of \cite{Warda2012a} that 180Hg deforms as a molecular system consisting of 90Zr and 72Ge, with the remaining neck nucleons being distributed at scission to give the fragments they found in the experiment. Similarly, they make the same claim for 198Hg, except using 98Zr and 80Ge. The first one kind of makes sense to me since 90Zr is semi-magic, but 98Zr is not and neither is 80Ge. I wonder what might have happened had they tried to match up the densities of a different set of nearby nuclei (they used these because they had the same N/Z ratio as the fissioning parent nucleus). Then in the conclusions: ``We conclude that the mass distribution of fission fragments in both nuclei is governed by shell structure of prescission configurations associated with molecular structures.''

Scamps and Simenel claim at the end of \cite{Scamps2018a} that perhaps the asymmetry can be explained by shell closures in octupole-deformed fragments, rather than spherical magic numbers.

In the introduction to \cite{Mcdonnell2014} it is stated as though conclusively that ``the main factor determining the mass split in fission are shell effects at pre-scission configurations, i.e., between saddle and scission'' (see also some additional references therein). I think the thing that is most selling it to me so far, though, is Fig. 3 from this paper, wherein they show the shell correction energy for each of the nuclei considered. Even though the PES itself is mostly flat in each of these cases, the magnitude of the shell correction is different whether you are looking at symmetric or asymmetric trajectories, and the one with the larger magnitude shell correction happens to be the one that wins out in the final fragment distribution. I'd also be curious to see what the collective inertia looks like, but this seems to at least give something. It's not like this shell correction gets added on top of the PES - the PES is still relatively-flat - but it at least gives an explanation for why our traditional physical intuition is not totally failing us here.

Interesting future work in this region might include calculations with full dynamics (including from nuclei with excitation energy), as suggested in the conclusions of \cite{Mcdonnell2014}






