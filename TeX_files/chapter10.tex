\chapter{List of my contributions}\label{append:Contributions}

\begin{itemize}
	\item \textbf{Observation of the competing fission modes in $^{178}$Pt}
	\begin{itemize}
		\item \textit{Phys. Lett. B 790, 583-588 (2019)}
		\item Performed PES and localization calculations
		\item Created graphics for Figure 3
		\item Suggested several revisions to the text of the article
	\end{itemize}
	\item \textbf{Colloquium: Superheavy elements: Oganesson and beyond}
	\begin{itemize}
		\item \textit{Rev. Mod. Phys. 91, 011001 (2019)}
		\item Generated Figure 12
		\item Helped write Section VI: Fission
		\item Suggested comments and revisions to the text
	\end{itemize}
	\item \textbf{Cluster radioactivity of $^{294}$Og}
	\begin{itemize}
		\item \textit{Phys. Rev. C 99, 041304 (2019)}
		\item Lead author on paper/wrote first draft
		\item Did all UNEDF1$_{\mathrm{HFB}}$ calculations and SkM* Langevin calculations
		\item Generated all figures
	\end{itemize}
%	\item \textbf{Microscopic Calculation of Fission Mass Distributions at Increasing Excitation Energies}
%	\begin{itemize}
%		\item \textit{Forthcoming CNR2018 Proceedings...}
%		\item Helped derive [one of the equations]
%		\item Contributed to discussion of ???
%	\end{itemize}
	\item \textbf{Talks}
	\begin{itemize}
		\item ``Cluster formation and emission in $^{294}$Og'', Nuclear Fission and Structure of Exotic Nuclei, \textit{March 2019}, Japan Atomic Energy Agency (JAEA), Tokai, Japan
		\item ``Fission in Exotic Nuclei Using Density Functional Theory'', FIRE Collaboration Meeting, \textit{May 2018}, North Carolina State University
		\item ``Cluster Emission in Oganesson-294'', Spontaneous and Induced Fission of Very Heavy and Super Heavy Nuclei Workshop, \textit{Apr 2018}, ECT*, Trento, Italy
		\item ``A `Microscopic' Description of Nuclear Fission'', Contemporary Physics Seminar, \textit{Sep 2016}, Indiana University South Bend
	\end{itemize}
	\item \textbf{Fission Tools}
	\begin{itemize}
		\item \textit{Suite of codes which extend the functionality of HFODD, HFBTHO, and other DFT solvers to the problem of nuclear fission.}
%		\item \href{https://gitlab.com/zachmath/fission_tools}{\verb|https://gitlab.com/zachmath/fission_tools|}
		\item \verb|https://gitlab.com/zachmath/fission_tools|
		\item Maintainer, 65 commits
		\item Converted old codes from Fortran77 to Fortran90
		\item Implemented shared memory (OpenMP) and distributed memory (MPI) parallelism
		\item Improved documentation, flexibility, and user-friendliness
		\item Created several Python-based utilities for plotting and file handling
		\item Increased functionality, such as by increasing from 2D to 3D or 4D
	\end{itemize}
	\item \textbf{PES Tools}
	\begin{itemize}
		\item \textit{Python framework for handling potential energy surfaces as XML files}
%		\item \href{https://gitlab.com/schuncknf/pes_tools}{\verb|https://gitlab.com/schuncknf/pes_tools|}
		\item \verb|https://gitlab.com/schuncknf/pes_tools|
		\item Developer, 25 commits
		\item Added a \verb|point| class with various methods to use on a single point of a PES
		\item Interface to \verb|fission_tools|
		\item Various updates bugfixes
	\end{itemize}
	\item \textbf{DFTNESS}
	\begin{itemize}
		\item \textit{DFTNESS (Density Functional Theory for Nuclei at Extreme ScaleS) is a computational framework to solve the equation of nuclear density functional theory. It is based on the two solvers HFBTHO and HFODD.}
		\item \verb|https://gitlab.com/schuncknf/dftness|
%		\item \href{https://gitlab.com/schuncknf/dftness}{https://gitlab.com/schuncknf/dftness}
		\item Developer, 60 commits
		\item OpenMP parallelization of subroutine \verb|QMULCM|
		\item Various updates and bugfixes
		\item NOTE: Most of these commits were related to \verb|fission_tools|, before that was all moved to a separate repository
	\end{itemize}
\end{itemize}