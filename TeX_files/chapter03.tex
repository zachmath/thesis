\chapter{Two fission modes in 178Pt}\label{chap:178Pt}

\maketitle

Fission is most well-studied in the region of the actinides (Z=90 to Z=103), as many naturally-occurring isotopes in this region are fissile. Within this region, there is a characteristic tendency for fission fragment yields to be asymmetric (that is, one light fragment and one heavy fragment), with the heavy peak centered around $A\approx140$. This has been understood as a manifestation of nuclear shell structure in the prefragments: doubly-magic $^{132}$Sn drives the nucleus towards scission, and once the neck nucleons are divided up between the two fragments, we end up with the heavy fragment A=140 peak. As one moves to the lower-Z actinides, however, this tendency becomes less and less pronounced as yields tend to become more symmetric. Below thorium, it was generally believed (though mostly not tested) that yields would continue to be symmetric as there was no doubly-magic nucleus candidate that could reasonably be expected to drive the system toward asymmetry as there is with actinides.

However, it was reported in a paper published in 20?? \cite{Andreyev20??} that neutron-deficient $^{180}$Hg undergoes beta-delayed fission to produce two fragments of unequal mass. This finding triggered a flurry of theoretical papers hoping to describe this new and unexpected phenomenon.

A follow-up experiment was performed in 20?? \cite{our-Pt-paper} investigating spontaneous fission of $^{178}Pt$, which differs from $^{180}$Hg by 2 protons. This system was studied at various excitation energies and found to fission pretty consistently with a bimodal pattern. Of the nuclei which underwent spontaneous fission, roughly 1/3 were found to fission symmetrically while the other 2/3 fissioned asymmetrically with a light-to-heavy mass ratio of approximately 80/98 (Is that the correct ratio?). Furthermore, it was observed that symmetric fragments tended to have higher kinetic energies than non-symmetric fragments. 



Elongation tends to minimize the Coulomb repulsion between fragments