\chapter{Temperature-Dependent ATDHFB Collective Inertia}\label{chap:TD-ATDHFB}

\maketitle

Everything which was shown in this dissertation assumed that the system was maintained at temperature $T=0$ and the nucleus behaved as a superfluid below the Fermi surface. However, in many environments (such as a neutron star merger or a nuclear blast) there may be quite a bit of excitation energy imparted to the system, which would raise the temperature above the Fermi surface. In this case, pairs may be broken and the topology of the potential energy surface may change (see, for instance, \cite{Mcdonnell2014}). In this case, the collective inertia of the system is changed, too, as shown below.