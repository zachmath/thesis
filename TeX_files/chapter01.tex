\chapter{Introduction}\label{chap:Intro}

\section{History of fission theory}
Nuclear fission is the fundamental physical process by which a heavy nucleus decays into two smaller nuclei of comparable masses, and a proper understanding of fission is critical for applications in reactor physics, nuclear astrophysics, and stockpile stewardship. Fission was first observed by Hahn and Stra\ss{}mann in 1939~\cite{Hahn1939} when they bombarded uranium atoms with neutrons and detected barium, a paradoxically lighter element, but at the time they were unable to explain their observations. An explanation came shortly thereafter in letters to Nature by Meitner and Frisch~\cite{Meitner1939b} and by Bohr~\cite{Bohr1939a}. Meitner and Frisch's described the findings in the following way:

\begin{quote}
On account of their close packing and strong energy exchange, the particles in a heavy nucleus would be expected to move in a collective way which has some resemblance to the movement of a liquid drop. If the movement is made sufficiently violent by adding energy, such a drop may divide itself into two smaller drops\dots \ It seems therefore possible that the uranium nucleus has only small stability of form, and may, after neutron capture, divide itself into two nuclei of roughly equal size (the precise ratio of sizes depending on finer structural features and perhaps partly on chance).
\end{quote}

\noindent In addition to neutron-induced fission, spontaneous fission, which is a different form of fission so-dubbed because it occurs without bombardment by neutrons or any other projectiles, was reported by Flerov and Petrjak in a single-paragraph letter to \textit{Physical Review} in 1940~\cite{Flerov1940}. For the remainder of this dissertation, I will be referring mainly to spontaneous fission unless stated otherwise.

%It is a highly-collective process involving all the constituent nucleons of the system, and thus since its discovery, it has been described via large shape deformations of an otherwise spherical (or nearly-spherical) ``drop'' of nucleons~\cite{Bohr1939}. 

Following the explanation of Meitner and Frisch, fission is relatively simple to conceptualize, but remarkably difficult to explain quantitatively. Quantitative fission predictions based on fundamental nuclear theory are challenging because of the large number of particles involved, along with the complex collective interactions which take place when one system deforms and becomes two. Historically, one could argue that theoretical attempts to describe nuclear fission have leapt forward in three major waves.

\subsection{Liquid drop model}
The first major wave in nuclear fission theory goes back to the very beginning of the nuclear age in the 1930s with the liquid drop model. The liquid drop model was first developed by Weizs\"acker in 1935~\cite{Weizsacker1935} as a way to describe collective properties of nuclei. It was later adapted by Bohr and Wheeler to quantitatively describe nuclear fission in terms of bulk properties of nuclei~\cite{Bohr1939}. This model was able to successfully describe nuclear binding energies and the energetics of nuclear fission.

The liquid drop model has its weaknesses, however. For example, it could not explain the fission fragment mass asymmetry which characterizes spontaneous fission in many actinides. Furthermore, until the 1960s, nuclear fission was treated as though separate from the rest of nuclear physics, with little attention given to the quantum nature of its constituent particles.%.. See the first few paragraphs of~\cite{Grant1976} for more of this story.

\subsection{Microscopic-macroscopic approach}
The second theoretical wave came during a time of renewed interest in fission, triggered by the discovery of fission isomerism in 1962 by Polikanov, et al~\cite{Polikanov1962}. This was understood as a manifestation of nuclear shape deformation based on a prediction by Nilsson~\cite{Nilsson1955}. Recognizing the important connection between shell effects, collective shape deformations, and the fission process, Strutinsky added a quantum mechanical correction to the liquid drop energy in 1967 in order to account for the added stability that occurs when a nucleus contains a ``magic number'' of protons and/or neutrons~\cite{Strutinsky1967, Strutinsky1968, Brack1972}. These models go by the name ``microscopic-macroscopic'' because they combine the ``macroscopic'' bulk properties of the liquid-drop model with the ``microscopic'' quantum mechanical Strutinsky shell correction.

Microscopic-macroscopic (``micmac'') fission models are computationally inexpensive, and can achieve quite satisfactory results (some recent highlights from the Los Alamos and Warsaw groups include Refs.~\cite{Moller2015a,Moller2015b,Jachimowicz2013,Jachimowicz2017}). In this approach, for instance, the mysterious fragment mass asymmetry that occurs in actinides is understood to be a manifestation of strong shell effects within the fragments during the split. However, since the model is based on a phenomenological description of what is actually a quantum many-body system, its predictive power is limited, with no clear way of making systematic improvements. Micmac models also rely on many parameters which must be fit to data, and are therefore subject to all the dangers that come with parameter fitting. A more fundamental approach would be to consider many individual nucleon states using a quantum many-body method. 

\subsection{Self-consistent models and the supercomputing era}
The third major wave is taking place now, heralded by the age of supercomputers. In fact, fission was listed as an exascale problem in a 2017 technical report to the Department of Energy~\cite{Carlson2017} - that is, one of the problems which motivates the drive towards exascale computing. For large systems with many, many particles, density functional theory (DFT) is a way to recast the ${\sim}$200 particle Schr\"{o}dinger equation into a simpler problem involving only a few densities and currents (see Section~\ref{sect:DFT} as well as~\cite{bender2003}). State-of-the-art calculations such as the ones described in this dissertation and others reviewed in~\cite{schunck2016} allow us to combine nuclear DFT techniques with modern high-performance computing platforms to predict fission properties, such as lifetimes and fragment yields.% Many of the ideas which are used today were inspired by previous work done with the liquid drop and micmac models. The approach used throughout this dissertation is described in Chapter~\ref{chap:Model}.

These advances in computing come simultaneously with technological advances which allow experimental nuclear physics to reach far beyond what has been achieved previously. New facilities include the Facility for Rare Isotope Beams (FRIB) at Michigan State University, which is projected to nearly double the number of isotopes which can be produced synthetically~\cite{Baumann2016}, and the Superheavy Elements Factory in Dubna, Russia~\cite{dmitriev2016}, which focuses on the synthesis of new superheavy elements and detailed study of superheavy isotopes that have already been created. Other, existing facilities have received, or will soon receive, upgrades which should allow them to produce experimental fission data that is much more detailed and precise than ever before~\cite{andreyev2018}. Together, state-of-the-art facilities for experiment and high-performance computing for theory are expected to lead to rapid advancement in our understanding of atomic nuclei and their decays.

\section{The scope of this dissertation}
Microscopic models (as self-consistent models are often called) are increasingly able to predict properties of fission fragments; however, a comprehensive microscopic description of fission fragments (including mass and charge distributions, excitation and kinetic energy distributions, angular dependence, spin, neutron emission) remains elusive. Chapter~\ref{chap:Model} will discuss our approach for estimating the mass and charge of fission fragments.

%Additionally, static models are not well-suited to describing the process of a single nucleus becoming two, an inherently time-dependent process. How can one precisely identify two distinct fragments when the wavefunctions of one fragment’s constituent nucleons may extend into the opposite fragment? How do correlations between nucleons affect the energetics of the resulting fragments? A better understanding of the mechanism of fragment formation can help guide and refine fission fragment models. This will be discussed in Section~\ref{sect:loc-frags}.

The next challenge is to do these calculations cheaply. In every theoretical calculation, one must ask the questions ``What approximations can I safely make?'' and ``What are the important degrees of freedom for this problem, and which can I ignore?'' These simplifications, together with improvements to the computational workflow itself, such as better file handling and parallelization, can significantly reduce the total time-to-answer. Computational details will be discussed in Chapter~\ref{chap:Numerical}.

After describing the model in Chapters~\ref{chap:Model} and~\ref{chap:Numerical}, it is applied to several isotopes in Chapters~\ref{chap:178Pt}-~\ref{chap:rprocess} to compute primary fragment yields. Historically, most experimental and theoretical studies of fission have centered around the region of actinides near $^{235}$U, which includes isotopes of uranium, plutonium, and thorium relevant for reactor physics and stockpile stewardship. Isotopes in this region tend to fission asymmetrically, with the larger prefragment influenced by the doubly-magic shell structure of $^{132}$Sn and resulting in a heavy fragment distribution centered around ${\sim}^{140}$Te. However, given the aforementioned recent interest in rare and exotic nuclei, we have applied our model to study fragment distributions in exotic systems found in other regions of the nuclear chart. First, in Chapter~\ref{chap:178Pt} we discuss bimodal fission in the neutron-deficient isotope {\Pt}, which until recently was conventionally expected to fission symmetrically. This region is a good one in which to test fission models because of the large isospin asymmetry ($N/Z\approx1.3$ in this region, compared to $N/Z\approx1.5$ near the valley of stability). Then in Chapter~\ref{chap:294Og} we discuss cluster radioactivity in {\Og}, the heaviest element ever produced in a laboratory. In Chapter~\ref{chap:rprocess} we move to the neutron-rich side of the nuclear chart ($N/Z>1.7$) to study isotopes which are expected to play a significant role in the astrophysical r process. %\textbf{Along the way, we will discuss some of the issues related to fragment identification and yield prediction.}

%The calculations in Chapters~\ref{chap:178Pt},~\ref{chap:294Og}, and~\ref{chap:rprocess} are relatively expensive. To perform large-scale exploratory studies in other regions of the nuclear chart, it will be necessary to find ways to reduce the total computational cost of these calculations. One method, still in its infancy, offers a promising approach for identifying fission fragment distributions using a significantly-reduced potential energy surface, which is by far the biggest bottleneck in our calculations. In Chapter~\ref{append:Fragments}, we discuss the problem of scission and present an alternative method for identifying fragments based on the nucleon localization function.

%Alternatively, at the end of each Chapter, we say a few words about challenges faced during the project and new physical insights gained that aren't related to the overall narrative of the Chapter, but which are nevertheless useful for future model developments.

This dissertation concludes in Chapter~\ref{chap:Outlook} with a discussion of our results and their significance. Suggestions are then made for future model developments, computational improvements, and physical applications.
