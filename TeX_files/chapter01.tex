\chapter{Introduction}\label{chap:Intro}

\section{History of Fission Theory}
Nuclear fission is the fundamental physical process by which a heavy nucleus decays to two smaller nuclei with approximately equal masses, and a proper understanding of fission is critical for applications in reactor physics, nuclear astrophysics, and stockpile stewardship. It is a highly-collective process involving all the constituent nucleons of the system, and thus since its discovery it has been described via large shape deformations of an otherwise spherical (or nearly-spherical) ``drop'' of nucleons. 

\subsection{Liquid Drop Model}
\subsection{Strutinsky shell correction}
\subsection{Self-consistent models and the supercomputing era}
[You should probably cite the fission discovery paper(s) by Hahn and Stra\ss{}mann \cite{Hahn1939} and the subsequent qualitative explanation paper by Meitner and Frisch \cite{Meitner1939}, not to mention the original liquid drop paper by Weizs\"acker \cite{Weizsacker1935} and the paper in which Bohr and Wheeler invoked the liquid drop model to describe fission quantitatively in terms of bulk properties of nuclei \cite{Bohr1939}. Finally, I might also do well to mention the spontaneous fission discovery paper \cite{Flerov1940}, which is actually just a letter to the editor of Physical Review that is only a paragraph long.] Models have grown increasingly sophisticated over time (the development of the Strutinsky shell correction to the LDM energy and the Funny Hills paper \cite{Strutinsky1967,Strutinsky1968,Brack1972}, which incorporated nuclear shell effects and explained the experimental observation that fragments were not equally-sized); however, the problem is incredibly complex. First of all, one must solve the nuclear many-body problem for a large (A>200) system. Then one must describe a transition from a single body to two. The way we do each of these things is described in Chapter \ref{chap:Model}.

If you're looking for a central narrative with which to tie together your thesis, you could, of course, use the whole ``making things faster'' angle you've been playing so far. But I think a more enriching, exciting, and satisfying approach would be to emphasize that you are doing fission calculations for \textit{rare} nuclei. That's cool because you work in a facility for \textit{rare} isotopes! And it's just one of those things that is interesting and fashionable in the field in general right now. The introduction to the platinum-178 paper has a good discussion about the importance of trying to understand fission in regions of exotic isospin ratios, and how simpler models tend to be less reliable in those regions. That covers both the platinum and the r-process project motivations (at least partially), and oganesson is just interesting because of how heavy it is.

It is an exciting time to study nuclear theory; major advances are now possible thanks to a groundwork lain of nuclear theory, paired with modern supercomputers fast enough for such complicated many-body problems to be solved.

It is an exciting time to study nuclear theory; for as we now enter the supercomputer age, we are able to implement the groundwork laid over the past several decades on modern, cutting-edge, high-performance computing centers. This allows 

These advances in computing come simultaneously with advances in accelerator design and technology and other advances which allow experimental nuclear physics to reach far beyond what has been done before. For instance, the Facility for Rare Isotope Beams (FRIB) at Michigan State University is projected to be able to nearly double the number of isotopes that can be produced synthetically. Together, state-of-the-art facilities for experiment and high-performance computing for theory are expected to lead to rapid advancement in our understanding of atomic nuclei.

One process which has always been a driver of nuclear physics is nuclear fission, the process by which a heavy nucleus decays into two smaller nuclei of approximately equal mass. Nuclear fission has been applied by humans in the fields of energy generation and national defense, and it has been predicted to play a major role in astrophysical environments such as neutron star mergers. There is currently a great deal of interest in understanding ``rare'' isotopes, or isotopes which are highly-unstable, in order to better understand such exotic phenomena as neutron star mergers, as well as to better identify physical properties which we can use to better understand the nuclei which we regularly encounter on Earth.

There are  a couple of ways for a nucleus to fission. One way is by imparting some excitation energy to a fissile nucleus, such as by bombarding a nucleus with neutrons (neutron-induced fission), by creating an excited nucleus as the decay product of another isotope (beta-delayed fission) or as a compound system of two collided nuclei. Owing to the randomness of quantum mechanics, another possibilty is for a nucleus in its ground state to spontaneously tunnel through a potential barrier and then emerge to form two distinct fragment (spontaneous fission). This dissertation will deal primarily/exclusively with the latter.

Fission, the fundamental process by which a single heavy nucleus splits into two smaller nuclei and a few emitted neutrons, is simple to understand qualitatively but remarkably difficult to explain quantitatively. One could argue that nuclear fission theory has leapt forward in three major waves. The first major wave of nuclear fission theory goes back to the very beginning of the nuclear age, when George Gamow proposed and Niels Bohr and John Archibald Wheeler developed the liquid drop model in the 1930s. This model was able to successfully describe nuclear binding energies and the energetics of nuclear fission. The second wave came with Strutinsky’s microscopic correction in the late 1960s, which essentially amounted to adding a quantum mechanical correction to the liquid drop energy. This correction, based on the nuclear shell model, is added in order to better account for the added stability that occurs when a nucleus contains a ``magic number'' of protons and/or neutrons \cite{Strutinsky1967, Strutinsky1968, Brack1972}. The third major wave is taking place now, heralded by the age of supercomputers. Now instead of using phenomenology and quantum corrections to describe heavy nuclei, we can use quantum many-body methods which were developed years or decades ago, but which were shelved until sufficiently-powerful computers came online in recent years.

Microscopic models (as they are called) are increasingly able to predict properties of fission fragments; however, a comprehensive description of fission fragments (including mass and fragment yields, excitation and kinetic energy distributions, angular dependence, spin, neutron emission) in a microscopic framework remains elusive. A major source of this elusiveness is due to the sheer difficulty of describing a smooth transition from one nucleus to two, a concept which is plagued with ambiguities. How can one precisely identify two distinct fragments when the wavefunctions of one fragment’s constituent nucleons may extend into the opposite fragment? And how do those correlations between nucleons affect the energeties of the resulting fragments? These are the questions to be addressed by this project.

Theoretically, making predictions about fission is challenging because, thanks to the large number of particles involved and the complex collective interactions which take place when one system deforms and becomes two, fission calculations have an "inextinguishable thirst for computing power," as stated in \cite{Schunck2016}. Historically, most fission calculations that have been done were based on empirical formulas or phenomenological models, most notably the ``microscopic-macroscopic'' family of models based on Bohr's liquid-drop model (to model the bulk properties of the nucleus) with Strutinsky shell corrections (to account for quantum mechanical shell effects). These microscopic-macroscopic (``micmac'') fission models are computationally fairly inexpensive, and can achieve quite satisfactory results. However, since the model is based on a phenomenological description of what is actually a quantum mechanical system, its predictive power is limited, and there is no clear way of making systematic improvements.

A more reliable approach would be to consider the individual nucleon states using some kind of quantum many-body method. For large systems with many, many particles, density functional theory (DFT) is a way to recast the Schr\"{o}dinger equation involving $\sim$200 particles into a simpler problem involving only a few densities and currents (see section \ref{sect:DFT}). With DFT as a way of calculating nuclear properties quantum-mechanically, one can then use these self-consistent solutions to predict fission properties, such as lifetimes and fragment yields. Fortuntately, a great deal of work has been done to achieve exactly this (see the review article \cite{Schunck2016}). Some of the ideas which are used were inspired by lessons learned from micmac and other, simpler models; others are unique to DFT. Our approach is described in detail in chapter \ref{chap:Model}.

The challenge, now, is to do these calculations cheaply. In every theoretical calculation, one must ask oneself ``What approximations can I safely make?'' and ``What are the important degrees of freedom for this problem?'' One may also reduce the total time-to-answer via improvements to the computational workflow itself, such as better file handling and parallelization.

\section{Goals of the project}
By far the most commonly-studied region so far for fission has been the region of actinides near $^{235}$U, which includes isotopes of uranium, plutonium, and thorium relevant for nuclear energy/reactor physics and stockpile stewardship/defense. Given the aforementioned recent interest in rare and exotic nuclei, we have applied our model to several exotic systems which undergo spontaneous fission in different regions of the nuclear chart. First in chapter \ref{chap:178Pt} we discuss bimodal fission in the neutron-deficient isotope platinum-178, which until recently was expected to fission symmetrically. Then in chapter \ref{chap:294Og} we discuss cluster radioactivity in oganesson-294, the heaviest element ever produced by humans. In chapter \ref{chap:rprocess} we move to the neutron-rich side of the nuclear chart to study ..., which are expected to play a major role in the astrophysical r-process. Finally, in chapter \ref{chap:Outlook} we discuss the current state of the field, and, based on our experience, offer insights for guiding future developments in the field.