\chapter{Introduction}\label{chap:Intro}

\section{History of fission theory}
Nuclear fission is the fundamental physical process by which a heavy nucleus decays to two smaller nuclei with comparable masses, and a proper understanding of fission is critical for applications in reactor physics, nuclear astrophysics, and stockpile stewardship. Fission was first observed by Hahn and Stra\ss{}mann in 1939 \cite{Hahn1939} as they bombarded uranium atoms with neutrons and detected barium, but the men could not explain their observations at the time. An explanation came shortly thereafter in letters to the editor by Meitner and Frisch \cite{Meitner1939b} and by Bohr \cite{Bohr1939a}. In Meitner and Frisch's paper they said ``On account of their close packing and strong energy exchange, the particles in a heavy nucleus would be expected to move in a collective way which has some resemblance to the movement of a liquid drop. If the movement is made sufficiently violent by adding energy, such a drop may divide itself into two smaller drops\dots It seems therefore possible that the uranium nucleus has only small stability of form, and may, after neutron capture, divide itself into two nuclei of roughly equal size (the precise ratio of sizes depending on finer structural features and perhaps partly on chance).'' A different form of fission, dubbed spontaneous fission because it occurs without bombardment by neutrons or any other projectiles, was reported by Flerov and Petrjak in a single-paragraph letter to \textit{Physical Review} in 1940 \cite{Flerov1940}. For the remainder of this dissertation, I will be referring mainly to spontaneous fission unless stated otherwise.

%It is a highly-collective process involving all the constituent nucleons of the system, and thus since its discovery, it has been described via large shape deformations of an otherwise spherical (or nearly-spherical) ``drop'' of nucleons \cite{Bohr1939}. 

Fission is relatively simple to conceptualize, but remarkably difficult to explain quantitatively. Making fission predictions based on fundamental nuclear theory is challenging because of the large number of particles involved, along with the complex collective interactions which take place when one system deforms and becomes two. Historyically, one could argue that theoretical attempts to describe nuclear fission have leapt forward in three major waves.

\subsection{Liquid drop model}
The first major wave of nuclear fission theory goes back to the very beginning of the nuclear age, with the liquid drop model in the 1930s. The liquid drop model was first developed by Weizs\"acker in 1935 \cite{Weizsacker1935} as a way of describing the collective properties of nuclei. It was later adapted by Bohr and Wheeler to quantitatively describe nuclear fission in terms of bulk properties of nuclei \cite{Bohr1939}. This model was able to successfully describe nuclear binding energies and the energetics of nuclear fission.

The liquid drop model had its weaknesses, however. For example, it could not explain the fission fragment asymmetry which characterizes spontaneous fission in many actinides. Furthermore, until the 1960s, nuclear fission was treated as though separate from the rest of nuclear physics.%.. See the first few paragraphs of \cite{Grant1976} for more of this story.

\subsection{Strutinsky shell correction approach}
The second theoretical wave came amidst a renewed interest in fission, triggered by the discovery of fission isomerism in 1962 by Polikanov, et al \cite{Polikanov1962}. This was understood as a manifestation of nuclear shape deformation based on a prediction by Nilsson \cite{Nilsson1955}. In 1967, Strutinsky added a quantum mechanical correction to the liquid drop energy in order to account for the added stability that occurs when a nucleus contains a ``magic number'' of protons and/or neutrons \cite{Strutinsky1967, Strutinsky1968, Brack1972}. These models go by the name ``microscopic-macroscopic'' because they combine the ``macroscopic'' bulk properties of the liquid-drop model with the ``microscopic'' quantum mechanical Strutinsky shell corrections.

These microscopic-macroscopic (``micmac'') fission models are computationally fairly inexpensive, and can achieve quite satisfactory results (some recent highlights from the Los Alamos and Warsaw groups include Refs. \cite{Moller2015a,Moller2015b,Jachimowicz2013,Jachimowicz2017}). However, since the model is based on a phenomenological description of what is actually a quantum many-body system, its predictive power is limited, and there is no clear way of making systematic improvements. A more reliable approach would be to consider individual nucleon states using some kind of quantum many-body method. 

\subsection{Self-consistent models and the supercomputing era}
The third major wave is taking place now, heralded by the age of supercomputers. Fission was listed in a recent 2017 report to the Department of Energy \cite{Carlson2017} as one of the problems which motivates the drive towards exascale computing. For large systems with many, many particles, density functional theory (DFT) is a way to recast the Schr\"{o}dinger equation involving $\sim$200 particles into a simpler problem involving only a few densities and currents (see Section \ref{sect:DFT}). With DFT as a way of calculating nuclear properties quantum-mechanically, one can then combine self-consistent methods with modern high-performance computing platforms to predict fission properties, such as lifetimes and fragment yields. Fortunately, a great deal of work has been done to achieve exactly this (see a recent review article on self-consistent approaches to fission in \cite{Schunck2016}). Some of the ideas which are used were inspired by lessons learned from micmac and other, simpler models; others are unique to DFT. One approach is described in detail in Chapter \ref{chap:Model}, and this is the model used to prepare this dissertation.

These advances in computing come simultaneously with advances in accelerator design and technology and other advances which allow experimental nuclear physics to reach far beyond what has been done before. For instance, the Facility for Rare Isotope Beams (FRIB) at Michigan State University is projected to be able to nearly double the number of isotopes that can be produced synthetically \cite{Baumann2016}. Together, state-of-the-art facilities for experiment and high-performance computing for theory are expected to lead to rapid advancement in our understanding of atomic nuclei and their decays.

\section{Predicting fission fragments}
Microscopic models (as self-consistent models are often called) are increasingly able to predict properties of fission fragments; however, a comprehensive description of fission fragments (including mass and fragment yields, excitation and kinetic energy distributions, angular dependence, spin, neutron emission) remains elusive in a microscopic framework. Chapter \ref{chap:Model} will discuss one approach for describing the mass and charge of fission fragments. The challenge now is to do these calculations cheaply. In every theoretical calculation, one must ask oneself ``What approximations can I safely make?'' and ``What are the important degrees of freedom for this problem?'' One may also reduce the total time-to-answer via improvements to the computational workflow itself, such as better file handling and parallelization.

Additionally, static models are not well-suited to describing the process of a single nucleus becoming two, in inherently time-dependent process. How can one precisely identify two distinct fragments when the wavefunctions of one fragment’s constituent nucleons may extend into the opposite fragment? And how do those correlations between nucleons affect the energetics of the resulting fragments? A better understanding of the mechanism of fragment formation can help guide and refine fission fragment models.


\section{Goals of the project}
Historically, most experimental and theoretical studies of fission have centered around the region of actinides near $^{235}$U, which includes isotopes of uranium, plutonium, and thorium relevant for reactor physics and stockpile stewardship. Isotopes in this region tend to fission asymmetrically, with the larger prefragment influenced by the shell structure of $^{132}$Sn and resulting in a heavy fragment distribution centered around $\sim^{140}$Te. However, recent experiments have highlighted other forms of fission which take place in other regions of the nuclear chart.

Given the aforementioned recent interest in rare and exotic nuclei, we have applied our model to study spontaneous fission in exotic systems found in other regions of the nuclear chart, with a focus on primary fragment yields. First, in Chapter \ref{chap:178Pt} we discuss bimodal fission in the neutron-deficient isotope {\Pt}, which until recently was conventionally expected to fission symmetrically. This region is a good one in which to test fission models because of the large isospin asymmetry ($N/Z\approx1.3$ in this region, compared to $N/Z\approx1.5$ near the valley of stability). Then in Chapter \ref{chap:294Og} we discuss cluster radioactivity in {\Og}, the heaviest element ever produced in a laboratory. In Chapter \ref{chap:rprocess} we move to the neutron-rich side of the nuclear chart ($N/Z>1.8$) to study isotopes which are expected to play a significant role in the astrophysical r process. %\textbf{Along the way, we will discuss some of the issues related to fragment identification and yield prediction.}

%The calculations in Chapters \ref{chap:178Pt}, \ref{chap:294Og}, and \ref{chap:rprocess} are relatively expensive. To perform large-scale exploratory studies in other regions of the nuclear chart, it will be necessary to find ways to reduce the total computational cost of these calculations. One method, still in its infancy, offers a promising approach for identifying fission fragment distributions using a significantly-reduced potential energy surface, which is by far the biggest bottleneck in our calculations. In Chapter \ref{append:Fragments}, we discuss the problem of scission and present an alternative method for identifying fragments based on the nucleon localization function.

%Alternatively, at the end of each Chapter, we say a few words about challenges faced during the project and new physical insights gained that aren't related to the overall narrative of the Chapter, but which are nevertheless useful for future model developments.

This dissertation concludes in Chapter \ref{chap:Outlook} with a discussion of results and their significance. Suggestions are then made for future model developments, computational improvements, and physical applications.
