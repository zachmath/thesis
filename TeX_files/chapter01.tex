\chapter{Introduction}

\maketitle
I don't really know how to make an abstract or something like that, and I know I'll have some other template to use when I actually start writing my thesis, but for the sake of having a place to put thoughts that may be useful later, here goes\dots

If you're looking for a central narrative with which to tie together your thesis, you could, of course, use the whole ``making things faster'' angle you've been playing so far. But I think a more enriching, exciting, and satisfying approach would be to emphasize that you are doing fission calculations for \textit{rare} nuclei. That's cool because you work in a facility for \textit{rare} isotopes! And it's just one of those things that is interesting and fashionable in the field in general right now. The introduction to the platinum-178 paper has a good discussion about the importance of trying to understand fission in regions of exotic isospin ratios, and how simpler models tend to be less reliable in those regions. That covers both the platinum and the r-process project motivations (at least partially), and oganesson is just interesting because of how heavy it is.