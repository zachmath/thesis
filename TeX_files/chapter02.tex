\chapter{Describing Fission Using Nuclear Density Functional Theory}\label{chap:Model}
\maketitle

(Nicolas gave a good annotated presentation in 2017 that describes some of the philosophy, as well as some of the outstanding challenges of spontaneous fission in an adiabatic framework: \verb|https://t2.lanl.gov/fiesta2017/school/Schunck_NotesSlides.pdf|)

Spontaneous nuclear fission is a type of quantum tunneling; consequently, it should be described using quantum mechanics.

\section{Nuclear Density Functional Theory}
Since nuclei are quantum mechanical systems, they can in principle be described using the Schrodinger equation. However, in practice one finds this type of description difficult or impossible, for two reasons:

\begin{itemize}
\item In order to use the Schrodinger equation, one needs to know how to describe the interaction between particles, such as between protons and neutrons. However, protons and neutrons are made up of quarks and gluons, which interact via the strong nuclear force. Consequently, an analytic expression for the nucleon-nucleon interaction analogous to the $\frac{1}{r}$ form of the Coulomb interaction is not available. Finding different mathematical expressions which can describe the interaction between nucleons continues to be an active area of research \cite{lots of papers}
\item Even when an interaction is known, nuclei are large systems made up of many protons and neutrons. Solving the Schrodinger equation directly quickly becomes computationally intractable as the number of nucleons increases.
\end{itemize}


\subsection{Density Functional Theory}\label{sect:DFT}
Kohn-Sham - Suppose you have the density of an interacting system. There exists a unique noninteracting system with the same density
Then I believe HFB is put on top of that to do the variation part. I think I (approximately) get it now! - So just to make sure, what would DFT look like without HF/HFB? And HF/HFB without DFT?

Rather than find the density of a system of interacting particles (which can be extremely complicated - as one particle moves, the force it exerts on neighboring particles causes them to move, which will in turn change the magnitude and direction of the net force acting on the original particle, and so on until an equilibrium configuration, if it exists, can be attained), Kohn-Sham allows us to find an equivalent density of fictional non-interacting particles. That is, instead of particles moving in a field generated by many interdependent neighboring particles, one may think of non-interacting particles moving about a mean-field, which is essentially an averaging over all other particles.

Together, the Hohenberg-Kohn theorems state that if one is able to find the true ground state density, regardless of where it comes from, then there exists a unique functional of the density which gives the ground state energy of the system. However, HK do not specify how this functional is to be obtained.

For a variety of reasons/complications, pure Kohn-Sham is not used in nuclear physics; however; in the spirit, we oftentimes switch to a representation involving densities (which are directly and exactly attainable from a many-body wavefunction) and energy density functionals (which are not known exactly). (Wait, but then what is the point of converting to densities? Why not just leave them as wavefunctions? Or maybe we do, but this representation just makes the math look nicer for papers)

\subsection{Skyrme Interaction}
In DFT there is proof of existence, but no recipe for finding the energy density functional corresponding to a given system. $\Rightarrow$ Is it true to say I'm actually using DFT? Or just SCMF via HFB? ...is only as good as its energy density functional. In principle the Kohn-Sham formulation of quantum mechanics is exact

We use DFT to recast many-body QM into something tractable (depending on $\rho$ and $\kappa$, not $\Psi$)
We use Skyrme as our approximate interaction (with coefficients fitted to data)
We use HF/HF+BCS/HFB to solve the Kohn-Sham equations by varying with respect to the configuration (in this case, $\rho$ and $\kappa$) (? This one I'm not as sure about)

The energy density means:

\begin{equation}
E = \int d^3\vec{r}\sum_{t=0,1}\mathcal{H}_t
\end{equation}

\noindent where $t=0(1)$ refers to isoscalar(isovector) energy densities. The total energy density is a sum of both time-even and time-odd terms:

\begin{eqnarray}
\mathcal{H}_t = \mathcal{H}^{even}_t + \mathcal{H}^{odd}_t \\
\mathcal{H}^{even}_t = C^\rho_t\rho_t^2 + C_t^{\Delta\rho}\rho_t\Delta\rho_t + C^\tau_t\rho_t\tau_t + C^J_t\mathsf{J}^2_t + C^{\nabla J}_t\rho_t\nabla\cdot\vec{J}_t \\
\mathcal{H}^{odd}_t = C^s_t \vec{s}_t^2 + C_t^{\Delta s}\vec{s}_t\Delta\vec{s}_t + C^T_t\vec{s}_t\cdot\vec{T}_t + C^j_t\mathsf{j}^2_t + C^{\nabla j}_t\vec{s}_t\cdot(\nabla\times\vec{j}_t)
\end{eqnarray}

\noindent where $\tau_t$ is the kinetic energy density; $\mathsf{J}_t$ is the spin current density, with vector part given by $\vec{J}_{\kappa,t} = \sum_{\mu\nu}\epsilon_{\mu\nu\kappa}\mathsf{J}_{\mu\nu,t}$; $\vec{s}_t$ is the spin density, $\vec{T}_t$ is the spin kinetic density; and $\vec{j}_t$ is the momentum density (to see how these are related to $\rho_t$, see, e.g., \cite{Bender2003}). Note that $\mathcal{H}^{even}_t$ depends only on time-even densities (and likewise for $\mathcal{H}^{odd}_t$).

\section{Microscopic Description of Nuclear Fission}
Should I describe these here, or in the sections when they actually get used?

With the nuclear physics somewhat under control, we now move onto the problem of using it to actually describe fission. For induced-fission, time-dependent density functional theory (TDDFT) allows one to calculate the time-evolution of a nucleus starting from some deformed initial configuration. So far, though, this approach has not been able to estimate a full yield for a fissioning nucleus; rather, the system propogates deterministically to a single scissioned configuration. Furthermore, and especially important for the case of spontaneous fission, the time-dependent approach does not allow for tunneling (why not?).

Nuclear fission is the fundamental physical process by which a heavy nucleus decays to two smaller nuclei with approximately equal masses, and a proper understanding of fission is critical for applications . It is a highly-collective process involving all the constituent nucleons of the system, and thus since its discovery it has been described via large shape deformations of an otherwise spherical ``drop'' of nucleons. In this framework, which is formalized by the adiabatic approximation, it falls upon theorists to describe many different nuclear shapes. In principle, one could describe any three-dimensional shape using an infinite basis such as the multipole expansion which is often encountered in eletrodynamics; however, for practical computations one must used a truncated set of only a few multipole moments (or, more generally, collective coordinates). Thus, an important challenge for researchers is to select the most relevant collective coordinates, ideally while demonstrating that others can be safely neglected.

Recently in \cite{Sadhukhan2016}, an approach based on this assumption was used to compute fragment yields from a potential energy surface (PES) that was computed self-consistently, using the WKB approximation to describe the tunneling and Langevin dynamics to describe post-scission dissipation. Now we test robustness of these results by exploring the impact of the energy density functional, the size of the collective space, and the calculation of the collective inertia on fragment yields.

\subsection{Potential Energy Surfaces}
Constrained HFB; map out many different constrained calculations and start to form a surface which resembles a topographical map. How do you decide which collective coordinates to use? Including pairing...

Physically, a nucleus can be thought of as a jumble of particles, each bouncing around in a potential well determined by the surrounding nucleons. Qunatum mechanics forbids us from determining the particle trajectories exactly; however, it allows us to estimate the probabilities of certain outcomes. In the case of fission, the most common approach has us thinking of nucleons grouping together collectively in a way which resembles a liquid drop (footnote: this idea was first proposed by Niels Bohr, I believe, and has proven to be a very fruitful way to describe fission). The collective shape is constrained to nearly-spherical shapes by a potential barrier; however, being a quantum mechanical system there is some nonzero tunneling probability, or a probability that the barrier will be penetrated, and the collective shape will stretch beyond the size fixed by the barrier. When this happens, the nucleus may remain in a long-term elongated state (called a fission isomer), or it may continue to deform until it separates into two fragments.

\subsection{WKB Approximation}

Adiabaticity: For fusion reactions, N,Z equilibrium reached in $\sim10^{-21}$ seconds, then energy/thermal equilibrium in a similar time scale, then finally mass equilibrium in $\sim10^{-19}$ - Yuri has a slide with these time scales from his talk Monday

Sort of an adiabatic approximation; useful because half-lives are long and therefore time-dependent approaches are impractical (they break down and/or become unstable or something after too many time steps, not to mention the amount of computing time). Wavefunction is assumed to be slowly-varying inside the potential barrier

Furthermore, TDHFB cannot tunnel

\subsection{Langevin Dynamics}

It has been shown (Jhilam and Nicolas' paper) that fission yields are fairly robust with respect to the dissipation strength
