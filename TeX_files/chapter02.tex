\chapter{Describing Fission Using Nuclear Density Functional Theory}\label{chap:Model}

%(Nicolas gave a good annotated presentation in 2017 that describes some of the philosophy, as well as some of the outstanding challenges of spontaneous fission in an adiabatic framework: \verb|https://t2.lanl.gov/fiesta2017/school/Schunck_NotesSlides.pdf|)

Today there are 2 microscopic approaches to spontaneous fission that are in common use: time-dependent and static (time-independent). Time-dependent approaches evolve the system in real-time. Since fission is an inherently time-dependent process, these methods offer great insight into the fission process and the characteristics of the fragments, especially kinetic and excitation energies \cite{Scamps2018, Scamps2015a, Simenel2014, Grineviciute2018, Umar2010}. However, they can only treat a single event at a time and are quite expensive, making them impractical for fission yield predictions. Despite efforts such as \cite{Scamps2015, Bulgac2018}, there is currently no way to obtain a full yield distribution in a time-dependent framework. Furthermore, time-dependent computations tend to break down after too many time steps, making them unsuitable for tunneling. This is a problem because spontaneous fission is fundamentally a quantum mechanical tunneling process.

On the other hand, static approaches assume that collective motions of the nucleus are slow compared to the motion of the intrinsic particles, and therefore that collective and intrinsic degrees of freedom can be decoupled. This assumption, called the adiabatic approximation, justifies the creation of a potential energy surface (PES) in some space of collective shape coordinates. The dynamics of fission are then described as trajectories across the PES. Calculating the kinetic and excitation energies in this framework is straightforward in principle, but in practice it is extremely sensitive to the scission configurations used. However, the static approach is much better suited to estimating fission yields and half-lives.

%Adiabaticity: For fusion reactions, N,Z equilibrium reached in $\sim10^{-21}$ seconds, then energy/thermal equilibrium in a similar time scale, then finally mass equilibrium in $\sim10^{-19}$ - Yuri has a slide with these time scales from his talk Monday. By comparison, what is an appropriate timescale for collective motion? I suppose that is nucleus-dependent

As we are trying to be as self-consistent as possible, we compute the PES in the framework of nuclear density functional theory, which combines the Hartree-Fock-Bogoliubov (HFB) mean-field approximation to the energy with a many-body method inspired by Kohn-Sham density functional theory (DFT). An overview of the self-consistent framework is described below, followed by the dynamical calculations which we use to calculate fission properties.

\section{Nuclear Density Functional Theory}
Since nuclei are quantum mechanical systems, they can in principle be described using the Schrodinger equation. However, in practice one finds this type of description difficult or impossible, for two reasons:

\begin{itemize}
\item In order to use the Schrodinger equation, one needs to know how to describe the interaction between particles, such as between protons and neutrons. However, protons and neutrons are made up of quarks and gluons, which interact via the strong nuclear force. Consequently, an analytic expression for the nucleon-nucleon interaction analogous to the $\frac{1}{r}$ form of the Coulomb interaction is not available. Finding different mathematical expressions which can describe the interaction between nucleons continues to be an active area of research \cite{lots of papers}
\item Even when an interaction is known, nuclei are large systems made up of many protons and neutrons. Solving the Schrodinger equation directly quickly becomes computationally intractable as the number of nucleons increases.
\end{itemize}

\subsection{Density Functional Theory}\label{sect:DFT}
Kohn-Sham DFT is based on the Hohenberg-Kohn theorems

Let us define the nucleon density in the following way: suppose we have a system described in second quantization by a set of creation and annihilation operators $c_i, c_i^\dagger$ which act on the [Fock-space?] vacuum state $\ket{\psi_0}$. The first Kohn-Sham theorem says that the energy of the system is a uniquely-defined functional of the density. That means that if a system of interacting particles and a system of noninteracting particles give the same density, the energy of those systems will be the same. This gives us the freedom to try to describe our system using a mean-field method instead of having to describe the pairwise interactions between every particle in the system - a huge simplification to the problem!

The second Kohn-Sham theorem states that the functional which gives the energy of the system will give the ground state energy if, and only if, it acts on the true ground state density. Thus, given a particular functional, we can vary the input density to minimize the total energy and be assured that we are approaching the ground state energy of the system.

Suppose you have the density $\rho(\mathbf{r})$ of an interacting system of particles. There exists a unique noninteracting system with the same density
Then I believe HFB is put on top of that to do the variation part. I think I (approximately) get it now! - So just to make sure, what would DFT look like without HF/HFB? And HF/HFB without DFT?

Rather than find the density of a system of interacting particles (which can be extremely complicated - as one particle moves, the force it exerts on neighboring particles causes them to move, which will in turn change the magnitude and direction of the net force acting on the original particle, and so on until an equilibrium configuration, if it exists, can be attained), Kohn-Sham allows us to find an equivalent density of fictional non-interacting particles. That is, instead of particles moving in a field generated by many interdependent neighboring particles, one may think of non-interacting particles moving about a mean-field, which is essentially an averaging over all other particles.

Together, the Hohenberg-Kohn theorems state that if one is able to find the true ground state density, regardless of where it comes from, then there exists a unique functional of the density which gives the ground state energy of the system. However, HK do not specify how this functional is to be obtained.

For a variety of reasons/complications (refs 73-78 of \cite{Schunck2016}), pure Kohn-Sham is not used in nuclear physics; however; in the spirit, we oftentimes switch to a representation involving densities (which are directly and exactly attainable from a many-body wavefunction) and energy density functionals (which are not known exactly). (Wait, but then what is the point of converting to densities? Why not just leave them as wavefunctions? Or maybe we do, but this representation just makes the math look nicer for papers)

The basic idea is to replace the single particle states $c_i^\dagger\ket{\psi_0}$ with the single-particle density, $\rho_{ij} = \expval{c_j^\dagger c_i}{\psi_0}$

Because pairing interactions are of great importance to nuclear dynamics, we also construct an additional density $\kappa_{ij} = \expval{c_jc_i}{\psi_0}$, which can be thought of as a coupling between the vacuum state and a state with two particles (in states $i$ and $j$). Together with the single-particle density $\rho$ we construct a generalized density

\begin{equation}
\mathcal{R} = \left(\begin{array}{cc}
\rho & \kappa \\
-\kappa^* & 1-\rho^*
\end{array}\right)
\end{equation}

In coordinate space, the density matrix $\rho$ and the pair tensor $\kappa$ take the form

\begin{align}
\rho(\vec{r},\vec{r}') &= \expval{c_{\vec{r}'}^\dagger c_{\vec{r}}}{\psi_0} \\
\kappa(\vec{r},\vec{r}') &= \expval{c_{\vec{r}'} c_{\vec{r}}}{\psi_0}
\end{align}

\noindent Recall that in nuclei, there is a $\rho_n$ describing neutrons and a $\rho_p$ for protons.

The total energy is a sum of several contributions:

\begin{equation}
E(\rho, \kappa) = E_{kin} + E_{Coul} + E_{nuc} + E_{pair}
%E(\rho, \kappa) = \int d^3\vec{r}\sum_{t=0,1}\mathcal{H}_t = \int d^3\vec{r}\left(\mathcal{H}_{kin} + \mathcal{H}_{nuc} + \mathcal{H}_{Coul, dir} + \mathcal{H}_{Coul, exch} + \mathcal{H}_{pair}\right)
\end{equation}

\noindent where $E_{kin}$ is the kinetic energy term, $E_{Coul}$ contains the Coulomb interaction between protons, $E_{nuc}$ is a phenomenological nucleon-nucleon interaction term, and $E_{pair}$ describes the tendency of nucleons to form pairs, which is smeared out in non-interacting mean-field models. Finding a good nucleon-nucleon interaction $E_{nuc}$ (and to a lesser extent, $E_{pair}$) to be used in calculations is an active topic of research in nuclear theory today (for one recent example, see \cite{NavarroPerez2018}); two types of interactions which are commonly-used today are the Skyrme and Gogny families of interactions \verb|\cite{???}|. We use primarily Skyrme-type interactions, which are described below.

\subsubsection{Kinetic term}

Defining the kinetic density $\tau_\alpha = \left.\nabla\cdot\nabla'\rho_\alpha(\vec{r},\vec{r}')\right|_{\vec{r}=\vec{r}'}$ , the kinetic energy contribution is

\begin{equation}
E_{kin} = \frac{\hbar^2}{2m} \left(1-\frac{1}{A}\right) \int d^3\vec{r} \left(\tau_n(\vec{r}) + \tau_p(\vec{r}) \right)
\end{equation}

\noindent The $\left(1-\frac{1}{A}\right)$ term is a simple, approximate center-of-mass correction.

\subsubsection{Coulomb interaction}
The Coulomb interaction between protons is divided into a direct term and an exchange term, which is related to the Pauli exclusion principle.

\begin{align}
E_{Coul} &= E_{Coul, dir} + E_{Coul, exch} \\
E_{Coul, dir}& = \frac{e^2}{2} \int d^3\vec{r}_1 d^3\vec{r}_2 \frac{\rho_p(\vec{r}_1)\rho_p(\vec{r}_2)}{\abs{\vec{r}_1-\vec{r}_2}} \\
E_{Coul, exch} &= \frac{e^2}{2} \int d^3\vec{r}_1 d^3\vec{r}_2 \frac{\rho_p(\vec{r}_2,\vec{r}_1)\rho_p(\vec{r}_1,\vec{r}_2)}{\abs{\vec{r}_1-\vec{r}_2}}
\end{align}

Often the exchange term is computed in the Slater approximation \verb|\cite{refs 27,28 of HFODD-I}|:

\begin{equation}
E_{Coul, exch} \approx -\frac{3e^2}{4} \left(\frac{3}{\pi}\right)^\frac{1}{3} \int d^3\vec{r} \rho_p^\frac{4}{3}(\vec{r})
\end{equation}

\subsubsection{Skyrme interaction}
The total Skyrme interaction energy density is a sum of both time-even and time-odd terms:

\begin{align}
E_{Skyrme} &= \int d^3\vec{r} \sum_{t=0,1} \left( \mathcal{H}^{even}_t + \mathcal{H}^{odd}_t \right)\\
\mathcal{H}^{even}_t &= C^\rho_t\rho_t^2 + C_t^{\Delta\rho}\rho_t\Delta\rho_t + C^\tau_t\rho_t\tau_t + C^J_t\mathsf{J}^2_t + C^{\nabla J}_t\rho_t\nabla\cdot\vec{J}_t \\
\mathcal{H}^{odd}_t &= C^s_t \vec{s}_t^2 + C_t^{\Delta s}\vec{s}_t\Delta\vec{s}_t + C^T_t\vec{s}_t\cdot\vec{T}_t + C^j_t\mathsf{j}^2_t + C^{\nabla j}_t\vec{s}_t\cdot(\nabla\times\vec{j}_t)
\end{align}

\noindent where $\tau_t$ is the kinetic energy density; $\mathsf{J}_t$ is the spin current density, with vector part given by $\vec{J}_{\kappa,t} = \sum_{\mu\nu}\epsilon_{\mu\nu\kappa}\mathsf{J}_{\mu\nu,t}$; $\vec{s}_t$ is the spin density, $\vec{T}_t$ is the spin kinetic density; and $\vec{j}_t$ is the momentum density (to see how these are related to $\rho$, see, e.g., \cite{Bender2003}). The index $t=0(1)$ refers to isoscalar(isovector) energy densities, e.g., $\rho_0 = \rho_n + \rho_p$ ($\rho_1 = \rho_n - \rho_p$). Note that $\mathcal{H}^{even}_t$ depends only on time-even densities (and likewise for $\mathcal{H}^{odd}_t$).

Since this interaction is phenomenological, based on a zero-range contact interaction between nucleons, the coefficients are adjustable. There are dozens of Skyrme parameterizations on the market, each one optimized to a particular observable or set of observables. The parameter sets SkM* \cite{Bartel1982} and UNEDF1 \cite{Kortelainen2012} (along with its sister, {\hfb} \cite{Schunck2015}) are optimized to datasets which include deformed nuclei, making them suitable for fission.

\subsubsection{Pairing interaction}
We use a density-dependent pairing interaction:

\begin{equation}
E_{pair} = V_0 \int d^3\vec{r} \left( 1-\left(\frac{\rho(\vec{r})}{\rho_0}\right)^\alpha \right)
\end{equation}

\noindent As with the nuclear interaction term, the pairing interaction contains several adjustable parameters.

\subsection{Bogoliubov transformation}

In anticipation of the HFB formalism below, we define the so-called Bogoliubov transformation. The fundamental entity in the Bogoliubov transformed basis are `quasiparticle' states, defined by quasiparticle creation and annihilation operators acting on a quasiparticle vacuum state $\ket{\Phi_0}$ (in contrast to the single particle operators from before). The creation and annihilation operators are given by

\begin{align}
\beta_\mu &= \sum_i U^*_{i\mu}c_i + \sum_i V^*_{i\mu}c_i^\dagger \\
\beta_\mu^\dagger &= \sum_i U_{i\mu}c_i^\dagger + \sum_i V_{i\mu}c_i
\end{align}

\noindent or in block matrix notation,

\begin{equation}
\left(\begin{array}{c} \beta \\ \beta^\dagger\end{array}\right) = 
\left(\begin{array}{cc} U^\dagger & V^\dagger \\ V^T & U^T \end{array}\right)
\left(\begin{array}{c} c \\ c^\dagger\end{array}\right)
\equiv \mathcal{W}^\dagger \left(\begin{array}{c} c \\ c^\dagger\end{array}\right)
\end{equation}

\noindent where the transformation matrix $\mathcal{W}$ must be unitary to ensure that $\beta, \beta^\dagger$ obey the fermion commutation relations \cite{Ring1980}. In this transformed basis, the density matrix takes the form 

\begin{equation}
\mathsf{R} = \mathcal{W}^\dagger\mathcal{R}\mathcal{W} = 
\left(\begin{array}{cc}
\expval{\beta_\mu^\dagger \beta_\nu}{\Phi_0} & \expval{\beta_\mu \beta_\nu}{\Phi_0} \\
\expval{\beta_\mu^\dagger \beta_\nu^\dagger}{\Phi_0} & \expval{\beta_\mu \beta_\nu^\dagger}{\Phi_0}
\end{array}\right) = 
\left(\begin{array}{cc}
0 & 0 \\
0 & I_N
\end{array}\right)
\end{equation}

%In general, $E$ is a functional of the generalized density $\mathcal{R}$.

\subsection{Hartree-Fock-Bogoliubov Equations}

The ground state configuration of the system described by this particular energy density functional $E$ is described by the density which minimizes $E(\mathcal{R})$. We can find this solution through the variational principle. We minimize the energy with respect to the generalized density, subject to the constraint that $\mathcal{R}^2=\mathcal{R}$, or in other words, that the state remains a quasiparticle vacuum. Defining the HFB Hamiltonian $\mathcal{H}_{ba} \equiv 2 \partial E/\partial \mathcal{R}_{ab}$, this variation leads to the result $\left[\mathcal{H},\mathcal{R}\right]=0$, which is called the Hartree-Fock-Bogoliubov equation. It is not typically solved in this form, but it can be recast into something more useful. Recalling that two Hermitian operators whose commutator is zero can be simultaneously diagonalized, we choose to diagonalize $\mathcal{H}$ using the same Bogoliubov transformation $W$ which diagonalizes $\mathcal{R}$:

\begin{equation}
W^\dagger \mathcal{H} W \equiv \mathcal{E} \qquad\mathrm{or}\qquad \mathcal{H}W = W\mathcal{E}
\end{equation}

\noindent where

\begin{equation}
\mathcal{E} = \left(\begin{array}{cc}
E_\mu & 0 \\
0 & -E_\mu
\end{array}\right)
\end{equation}

\noindent is a matrix of quasiparticle energies. In this form, the problem can then be solved iteratively: an initial density ansatz is chosen in order to construct the Hamiltonian density $\mathcal{H}$, after which the eigenvalue problem is solved, leading to new densities (since the densities are related to $\mathcal{W}$), which in turn leads to an updated $\mathcal{H}$. This procedure can be repeated indefinitely, until some predetermined convergence criterion is met.
% https://ocw.mit.edu/courses/physics/8-04-quantum-physics-i-spring-2013/study-materials/MIT8_04S13_OnCommEigenbas.pdf

Very often we will want to minimize the energy with the system subject to a particular constraint. In that case we would replace the Hamiltonian $E$ with the Routhian $E'$ before variation. Typically $E'$ introduces the constraints via the method of Lagrange multipliers. Some common examples might be this simple form of particle number restoration (more sophisticated forms, such as Lipkin-Nogami \verb|\cite{???}|, also exist)

\begin{equation}
E' = E - \lambda_n \expval{\hat{N}_n} - \lambda_p \expval{\hat{N}_p}
\end{equation}

\noindent where $\lambda_\alpha$ is determined later by the condition that $\expval{\hat{N}_\alpha} = N_\alpha$, or shape, where we might constrain a particular multipole moment (or set of multipole moments) to the value $\bar{Q}_{\lambda\mu}$

\begin{equation}
E' = E - \sum_{\lambda\mu} C_{\lambda\mu} \left(\expval{\hat{Q}_{\lambda\mu}} - \bar{Q}_{\lambda\mu}\right)^2
\end{equation}


\subsection{Nucleon localization function}\label{sect:locali}
One of the tools we will be using quite a bit in this thesis is the nucleon localization function (NLF), introduced in \cite{Zhang2016}. The NLF is defined using the single particle density in the following way (with q=isospin and $\sigma$=spin/signature quantum number):

\begin{equation}
\mathcal{C}_{q\sigma} = \left[1+\left(\frac{\tau_{q\sigma}\rho_{q\sigma}-\frac{1}{4}|\nabla\rho_{q\sigma}|^2-\mathbf{j}^2_{q\sigma}}{\rho_{q\sigma}\tau_{q\sigma}^{TF}}\right)^2\right]
\end{equation}

\noindent where $\tau_{q\sigma}^{TF}=\frac{3}{5}(6\pi^2)^\frac{2}{3}\rho_{q\sigma}^\frac{5}{3}$. A localization value $\mathcal{C} \approx 1$ means that nucleons are well-localized; that is, the probability of finding two nucleons of equal spin and isospin at the same location in space is low. A value of $\mathcal{C}=\frac{1}{2}$ corresponds to a Fermi gas of nucleons, as found in nuclear matter.

The NLF offers greater insight into the underlying shell structure of the system than, for instance, the single particle density. In particular, when applied to fission as in \cite{Sadhukhan2017}, it sometimes enables one to see the formation of well-defined prefragments whose shell structure is responsible for the peak of the fragment distribution. A method for identifying fission fragments and estimating fragment distributions using the NLF is described in Appendix \ref{append:Fragments}.

\section{Microscopic Description of Nuclear Fission}
With the nuclear physics somewhat under control, we now move onto the problem of using it to describe fission. Recently in \cite{Sadhukhan2016}, an approach based on this assumption was used to compute fragment yields from a potential energy surface (PES) that was computed self-consistently, using the WKB approximation to describe the tunneling and Langevin dynamics to describe post-scission dissipation. The half-life can be computed as in \cite{Sadhukhan2013}.

\subsection{Potential Energy Surfaces}
In the adiabatic approximation, the primary degrees of freedom are nuclear shapes, and therefore the basic ingredient to fission calculations is a potential energy surface (PES). In principle, one could describe any three-dimensional shape using an infinite basis such as the multipole expansion which is often encountered in electrodynamics; however, for practical computations one must used a truncated set of only a few collective coordinates. Thus, an important challenge for researchers is to select the most relevant collective coordinates, ideally while demonstrating that others can be safely neglected. Often one will use the first few lowest-order multipole moments; however, multipole moments may not always be well-suited to describing shapes which occur during fission, especially near scission. One alternative was proposed in \cite{Younes2012}.

Once the appropriate shape constraints are chosen, the PES is computed as a mesh: one DFT calculation per grid point. The value at each point is the HFB energy computed above, $E'(\vec{q})$.

\subsection{Collective inertia}
Just as important to the fission dynamics as the energy of the system is the collective inertia, which describes the tendency of the system to resist configuration changes (such as shape changes). The form of the collective inertia we use is the non-perturbative adiabatic time-dependent HFB (ATDHFB) inertia with cranking \cite{Baran2011}, which takes the form

\begin{equation}\label{eq:mATDHFB-np}
\mathsf{M}_{\mu\nu} =  \frac{\hbar^2}{2}\frac{1}{(E_a+E_b)}\left(\frac{\partial\mathcal{R}^{21}_{(0),ab}}{\partial q_\mu}\frac{\partial\mathcal{R}^{12}_{(0),ba}}{\partial q_\nu}+\frac{\partial\mathcal{R}^{12}_{(0),ab}}{\partial q_\mu}\frac{\partial\mathcal{R}^{21}_{(0),ba}}{\partial q_\mu}\right)
\end{equation}

\noindent The subscripts and superscripts are explained in the full temperature-dependent derivation of the collective inertia found in Appendix \ref{append:TD-ATDHFB}, but the important feature to note is that computing the inertia requires differentiating the density matrix with respect to a set of collective coordinates.

A perturbative expression for the ATDHFB inertia also exists, which allows one to estimate the inertia without taking derivatives of the density. It is computationally much faster and easier to implement, but it is less accurate and loses many of the important features of the inertia, as we shall see in Chapter \ref{chap:294Og}. Nevertheless, it is commonly-used in calculations and we shall use it later on.

Another common expression for the collective inertia comes from the Generator Coordinate Method (GCM). The GCM inertia also exists in two varieties: perturbative and non-perturbative \cite{Giuliani2018b}. Like the ATDHFB inertia, the perturbative GCM inertia is smoothed-out compared to the non-perturbative inertia. Both the perturbative and non-perturbative GCM inertias are found to be smaller in magnitude than their ATDHFB counterparts.

\subsection{WKB Approximation}\label{sect:wkb}
Spontaneous nuclear fission is a type of quantum tunneling; consequently, it should be described using quantum mechanics. If the wavefunction corresponding to the fissioning nucleus is assumed to be slowly-varying inside the potential barrier (which is the case under the adiabatic assumption), then the WKB approximation allows us to estimate the tunneling probability through a classically-forbidden region in the PES.

Consider a set of collective coordinates $\mathbf{q}\equiv(q_1, \ldots, q_N)$. The most-probable tunneling path $\left. L(s) \right|_{s_{\rm in}}^{s_{\rm out}}$ in the collective space is found via minimization of the collective action

\begin{equation}\label{eq:action} 
S(L) = \frac{1}{\hbar}\int_{s_{\rm in}}^{s_{\rm out}} \sqrt{2\mathcal{M}(s)\left(V(s)-E_0\right)}ds,
\end{equation} 

\noindent where $s$ is the curvilinear coordinate along the path $L$,
$\mathcal{M}(s)$ is the collective inertia given by \cite{Sadhukhan2013}

\begin{equation}
\mathcal{M}(s) = \sum_{\mu\nu} \mathsf{M}_{\mu\nu} \frac{dq_\mu}{ds} \frac{dq_\nu}{ds}
\end{equation}

\noindent and $V(s)$ is the potential energy along $L(s)$. $E_0$ stands for the collective ground-state
energy. The dynamic programming method \cite{Baran1981} is employed to determine
the path $L(s)$. The calculation is repeated for different outer turning points,
and each of these points is then assigned an exit  probability $P(s_{\rm out})=[1+\exp{(2s)}]^{-1}$ \cite{Baran1978}. 

The half-life corresponds to the minimum action pathway, and the expression for the half-life is $T_{1/2} = \mathrm{ln}(2)/nP(s_\mathrm{min})$. The parameter $n$ is the number of assaults on the fission barrier per unit time and the standard value is $n=10^{20.38} s^{-1}$.

\subsection{Langevin Dynamics}
Two techniques for predicting fission fragment yields are the Langevin approach used by \cite{Sadhukhan2016} and the Time-Dependent Generator Coordinate Method (TDGCM) approach used in \cite{Younes2012}. I will be using Langevin dynamics, which are described in this section.

After emerging from the classically-forbidden region of the PES, fission trajectories begin from the outer turning
line and then evolve along the PES according to the Langevin equations:
\begin{gather}\label{eq:langevin} 
	\frac{dp_i}{dt} =  
	-\frac{p_j p_k}{2} \frac{\partial}{\partial q_i}\left(\mathcal{M}^{-1}\right)_{jk} 
	- \frac{\partial V}{\partial q_i}  - \eta_{ij}\left(\mathcal{M}^{-1}\right)_{jk} p_k + g_{ij}\Gamma_j(t) \,, \\ 
	\frac{dq_i}{dt} = 	\left(\mathcal{M}^{-1}\right)_{ij} p_j \,,  
\end{gather} 
where $p_i$ is the collective momentum conjugate to $q_i$. The dissipation
tensor $\eta_{ij}$ is related to the random force strength $g_{ij}$ via the
fluctuation-dissipation theorem, and $\Gamma_j(t)$ is a Gaussian-distributed,
time-dependent stochastic variable.

The fluctuation-dissipation theorem is given by the expression $\sum_k g_{ik}g_{jk} = \eta_{ij}k_BT$. It effectively couples the collective and intrinsic via the system temperature, given by $k_BT = \sqrt{E^*/a}$ where $a=A/10$MeV$^{-1}$ parameterizes the level density and the excitation energy $E^* = V(s_{out}) - V(\mathbf{x}) - \frac{1}{2}\sum\left(\mathcal{M}^{-1}\right)_{ij}p_i p_j$.

Dissipation is treated in our work as a parameter, as a self-consistent description of dissipation is not yet known. However, work along this line has been started (maybe?) in refs 291-293 of \cite{Schmidt2018} (see section 4.1.1 for the context). In the meantime, we use the values from \cite{Sadhukhan2016} (Is this too specific for a thesis? You're not worried about the little numerical details, right? Just the big-picture ideas?)

