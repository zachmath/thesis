\chapter{Nuclear Density Functional Theory}

\maketitle
Since nuclei are quantum mechanical systems, they can in principle be described using the Schrodinger equation. However, in practice one finds this type of description difficult or impossible, for two reasons:

\begin{itemize}
\item In order to use the Schrodinger equation, one needs to know how to describe the interaction between particles, such as between protons and neutrons. However, protons and neutrons are made up of quarks and gluons, which interact via the strong nuclear force. Consequently, an analytic expression for the nucleon-nucleon interaction analogous to the $\frac{1}{r}$ form of the Coulomb interaction is not available. Finding different mathematical expressions which can describe the interaction between nucleons continues to be an active area of research \cite{lots of papers}
\item Even when an interaction is known, nuclei are large systems made up of many protons and neutrons. Solving the Schrodinger equation directly quickly becomes computationally intractable as the number of nucleons increases.
\end{itemize}

\subsection{Skyrme Interaction}

\subsection{Density Functional Theory}

\section{Microscopic Description of Nuclear Fission}
Should I describe these here, or in the sections when they actually get used?

\subsection{Potential Energy Surfaces}
Constrained HFB; map out many different constrained calculations and start to form a surface which resembles a topographical map

\subsection{WKB Approximation}

Sort of an adiabatic approximation; useful because half-lives are long and therefore time-dependent approaches are impractical (they break down and/or become unstable or something after too many time steps, not to mention the amount of computing time). Wavefunction is assumed to be slowly-varying inside the potential barrier

Furthermore, TDHFB cannot tunnel

\subsection{Langevin Dynamics}

It has been shown (Jhilam and Nicolas' paper) that fission yields are fairly robust with respect to the dissipation strength