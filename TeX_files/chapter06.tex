\chapter{Outlook}\label{chap:Outlook}
\maketitle

In this chapter, it would be great to talk to everyone you know (Witek, Samuel, Jhilam, Nicolas, Michal, and so on) to get a better feel for what kinds of issues need to be addressed next. You've already got sort of a rudimentary understanding (see your Google Keep note for starters), but it might be good to get some outsider perspective. This will be especially important as you start looking for postdocs, and \textit{especially} especially if you end up looking for postdocs in nuclear theory, but not necessarily nuclear fission.

As I said in chapter \ref{chap:Intro}, ``Finally, in chapter \ref{chap:Outlook} we discuss the current state of the field, and, based on our experience, offer insights for guiding future developments in the field.''

We definitely need a better handle on the inertia. The perturbative inertia is easy to compute, but not terribly reliable. The non-perturbative inertia can certainly do better, but as it is computed now (using finite differences) it is subject to numerical artifacts and instabilities (dependent on the level of convergence of the individual densities, the coefficient mutlipliers, different basis sizes) and actual physics, such as level crossings which manifest in projections from a higher-dimensional space.

UNEDF1 seems to underestimate fission barrier heights (artificial though the concept may be; the main impact is probably that lifetimes are underestimated). It also turns out to be a headache to work with, making convergence quite a challenge sometimes (any cases in particular, like for highly-deformed or heavy or octupole-deformed nuclei or something?). Better functionals might hope to better capture the physics, and one can hope they are easier to work with.