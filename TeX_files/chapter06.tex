\chapter{Outlook}\label{chap:Outlook}

\section{Insights gained}
\subsection{Prefragment shell stsructure}
A common theme in all of this has been the importance of the underlying shell structure of the prefragments. Shell energy corrections were found to be important in {\Pt} and {\Hg}; cluster formation in {\Og} was clearly influenced by the shell structure of the fragments; and the same may or may not be the case for {\Cf}. Let's discuss this.

We used localizations to visualize the internal/intrinsic shell structure inside nuclei, and we were able to see that this structure was sometimes intact early in the evolution, at times as far back as the outer turning line. And actually, this kind of makes sense. From just energetics alone, a nucleus on the outer turning line is just as happy (or just as stable, or just as settled) as a nucleus in the ground state. In some sense, it is formed. The difference now is just that the configuration it's in is now unstable due to Coulomb. The two halves, which are kind of maybe happy from a nuclear physics perspective, are pushing apart from the Coulomb repulsion. So that still has to be carried out, but the bulk of the physics might already be done at this point - though not necessarily. It \textit{could} be that the fragments are well-formed and just pushing apart, but that may not be the case. It's like a divorce: sometimes the two have drifted so far apart, or are so well-defined and incompatible as individuals that the divorce is simple and relatively straightforward. Other times, it is a mess trying to sort out who gets what, and the two parties are fundamentally-changed by the proceedings.

I don't have any strong objections to Scamps and Simenel's octupole paper. In fact, to me it kind of makes sense: we've been saying, after all, that it's the shell structure of the deformed prefragments which determine scission, and not necessarily the final fragments themselves. That's really the whole idea behind the localization paper: we're seeing that, at least in some cases, the shell structure is pretty well intact early in the evolution, and that those prefragments drive the system to scission with some shuffling of the neck nucleons at scission. All they're saying is that those neck nucleons will effect the shell structure of the prefragments, and just based on the kinds of shapes that the system will take (small neck connecting two elongated or spherical fragments), the prefragments have a strong octupole moment (regardless of whether the fragments are elongated or spherical). So it shouldn't be the spherical magic numbers we worry about, but the deformed (in this case, octupole-deformed) magic numbers.

I feel like it shouldn't be too terrible to investigate this claim. What if we constrained the multipole moment(s) that correspond(s) to octupole-deformed fragments (perhaps $Q_{50}$)? I think this parameter might be included in Peter Moller's model, but not in ours.

\section{Review, outlook, and perspectives}

In this chapter, it would be great to talk to everyone you know (Witek, Samuel, Jhilam, Nicolas, Michal, and so on) to get a better feel for what kinds of issues need to be addressed next. You've already got sort of a rudimentary understanding (see your Google Keep note for starters), but it might be good to get some outsider perspective. This will be especially important as you start looking for postdocs, and \textit{especially} especially if you end up looking for postdocs in nuclear theory, but not necessarily nuclear fission.

As I said in chapter \ref{chap:Intro}, ``Finally, in chapter \ref{chap:Outlook} we discuss the current state of the field, and, based on our experience, offer insights for guiding future developments in the field.''

At this stage, we have techniques to calculate half-lives and primary fragment distributions (I haven't mentioned it yet, but there is also Nicolas' method for the fragment yields that uses TD-GCM. Are there others? What about for half-lives?). Some methods (such as Walid's, TDDFT, and possibly also this GCM method) are starting to estimate fragment energetics (kinetic and excitation energies). Down the line, there are others who try to predict neutron multiplicities and goodness knows what else using Hauser-Feshbach models and such (FREYA and more). These regions are still disconnected. Of course, these methods still need major refinements in order to better reflect experimental data. Some ideas currently in the pipeline for improving the models are:

\begin{itemize}
	\item Improved EDFs (here you could mention the DME EDFs)
	\item Improved inertia tensor (such as automatic differentiation)
	\item Better/more collective coordinates (Walid's $D, \xi$ coordinates or whatever they were called [Technical Report LLNL-TR-586678 (2012) Fragment yields calculated in a time-dependent microscopic theory of fission]; continuity of the PES in $\infty$-dimensional space such like in David Regnier's talks and papers)
	\item Fragment identification (our localization paper, Marc Verriere's method; you might also mention that this is not an issue in TDDFT, but there you've only got one single fragment pair)
	\item Microscopic/self-consistent description for dissipation. This is the mechanism which exchanges between intrinsic and collective degrees of freedom, but we handle it in a very ad hoc way with parameters which are fitted instead of determined systematically through some theory. Solving this problem will probably help us with the energetics of fragments (TKE and E* at the same time!)
\end{itemize}

Furthermore, there are more experimental observables that we should try to predict (refer to Andreyev's review to see what other observables can currently be measured). These include energetics (TKE and E*, for we have only begun to scratch the surface here), angular momentum, prompt neutron multiplicities (is that within the scope of these self-consistent models?), prompt neutron and gamma energy spectra spectra (getting harder; these are usually handled via statistical models; see intro to \cite{Schmidt2018} for some references), level densities?, and probably more but my mind is blanking. How to compute these in a self-consistent framework is still an open question. See also the outlook in Nicolas' review.

We definitely need a better handle on the inertia. The perturbative inertia is easy to compute, but not terribly reliable. The non-perturbative inertia can certainly do better, but as it is computed now (using finite differences) it is subject to numerical artifacts and instabilities (dependent on the level of convergence of the individual densities, the coefficient mutlipliers, different basis sizes) and actual physics, such as level crossings which manifest in projections from a higher-dimensional space.

UNEDF1 seems to underestimate fission barrier heights (artificial though the concept may be; the main impact is probably that lifetimes are underestimated). It also turns out to be a headache to work with, making convergence quite a challenge sometimes (any cases in particular, like for highly-deformed or heavy or octupole-deformed nuclei or something?). Better functionals might hope to better capture the physics, and one can hope they are easier to work with.