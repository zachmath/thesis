\chapter{Finite-temperature ATDHFB collective inertia}\label{append:TD-ATDHFB}



%Be sure to discuss the complications which arise in the finite temperature formalism, as promised in Chapter~\ref{chap:Numerical}. In essence, you end up dividing by terms which are very small. You can avoid dividing by zero by introducing a cutoff. If the cutoff is too large, you lose some of the data in the tail. If the cutoff is too small, you divide by numbers that are smaller than the noise in the density. There are actual numbers in your dudeman4 Google Drive, in a file called Inertia Tensor Convergence.

The calculations presented in this dissertation assumed that the system was maintained at temperature $T=0$ and that the nucleus behaved as a superfluid. However, in future calculations, in order to properly describe fission from an excited compound nucleus as in the case of Chapter~\ref{chap:178Pt}, or neutron-induced fission as in Chapter~\ref{chap:rprocess}, one should technically use a finite-temperature formalism.

In the $T\neq0$ case, pairs may be broken and various excited states are accessible, resulting in changes to the PES and the collective inertia. A finite-temperature formalism for calculating the PES has already been developed (see~\cite{Schunck2015b, Mcdonnell2014}), as have perturbative expressions for the finite-temperature adiabatic time-dependent Hartree-Fock-Bogoliubov (FT-ATDHFB) inertia with cranking~\cite{Egido1986, Martin2009, Schunck2019edf}. Since the yields in Chapter~\ref{chap:294Og} showed a dependence on whether the collective inertia was computed perturbatively or non-perturbatively, this chapter is dedicated to deriving a non-perturbative finite-temperature expression for the collective inertia.

We will begin with a brief overview of ATDHFB theory and finite temperature HFB. Then we present the resulting collective inertia without detailed derivation, which is straightforward.


%### How much of the derivation mirrors Nicolas'? Should we just briefly introduce ATDHFB, finite-temp, and then just present the cranking expression without any real derivation? That kind of defeats the purpose of what you were thinking but as long as it makes Witek happy...###

%The idea of considering fission as a finite temperature process stems from trying to develop a picture of induced fission. In induced fission, a neutron which carries some amount of energy is captured by a heavy nucleus in its ground state. That extra energy's gotta go somewhere, but where? In a large nucleus, you have all sorts of places, including any number of single particle excitations, or combination of single particle excitations, or perhaps the entire nucleus moves together as one large collective excitation. Apparently some people went through and did the combinatorics of these possible excitations and decided that the number of them was huge (like, ${\sim}10^{12}$ huge)~\cite{Hilaire2012}. So handling them explicitly just isn't going to work.
%
%Additionally, DFT might not be the best tool for performing finite-temperature calculations. This is because DFT is typically implemented as a variational method, which means it's good for ground state calculations. But highly-deformed nuclei are inherently \textit{not} in their ground state. In fact, this is true even for ``zero-temperature" DFT as well. In practice the results we get are pretty good (most likely the system has time to equilibrate to its ``deformed ground state" at each deformation step, or in other words the path to scission proceeds ${\sim}$adiabatically), but it's something to definitely keep in mind that (I suspect) could strongly affect your half-life predictions and fragment energies in particular. Furthermore, it's not clear to me how you might correct that should the need arise.
%
%A third thing that Nicolas claims is that the temperature should depend on the actual deformation. I'm not sure where exactly this comes from but it sounds plausible to me.


%A lot of these ideas I'm getting from~\cite{Schunck2015b} as well as Nicolas' own temperature-dependent HFB notes.


\section{Adiabatic time-dependent HFB}

Adiabatic time-dependent Hartree-Fock, which is generalized to adiabatic time-dependent Hartree-Fock-Bogoliubov, is presented carefully in~\cite{engel1975, Baranger1978}; we merely summarize the relevant results here. The fundamental assumption of Time-Dependent Hartree-Fock-Bogoliubov (TDHFB) theory is that a system is described by a Slater determinant. This assumption allows us to write the TDHFB equation:
\begin{equation}\label{eq:TDHFB}
i\hbar \mathcal{\dot{R}} = \left[\mathcal{H},\mathcal{R}\right],
\end{equation}

\noindent where
\begin{equation}
\mathcal{\tilde{H}} = 
\left(\begin{array}{cc}
h-\lambda & \Delta \\
-\Delta^* & -h^*+\lambda
\end{array}\right), 
\qquad \mathcal{\tilde{R}} = 
\left(\begin{array}{cc}
\rho & \kappa \\
-\kappa^* & 1-\rho^*
\end{array}\right).
\end{equation}

One \textit{additional} assumption, which has already been discussed in Chapter~\ref{chap:Model}, is that collective motion is slow compared to single particle motion of the system. This is called the adiabatic approximation, and the consequent model is called Adiabatic Time-Dependent Hartree-Fock-Bogoliubov (ATDHFB). Both here and in Chapter~\ref{chap:Model}, the adiabatic approximation is invoked in order to describe the dynamics of a fissioning nucleus in terms of a small set of collective variables and their derivatives (analogous to coordinates and velocities): suppose there is a set of collective variables $(q_1,q_2,\dots,q_n)$ which describes changes in the density $\mathcal{R}(t)$ at all times $t$. Then
\begin{equation}\label{eqn:RdotQdot}
\mathcal{\dot{R}} = \sum_{\mu=1}^{n}\dot{q_\mu}\frac{\partial\mathcal{R}}{\partial q_\mu}.
\end{equation}

Once the system is described in terms of collective coordinates and velocities, the energy can be expressed as the sum of a ``potential" term (which depends on the coordinates) and a ``kinetic" term (which depends on the velocities). Our goal is to understand the kinetic part of the energy, which in some sense describes the dynamics of a deformed nucleus. A key component of this will be the inertia tensor $\mathcal{M}$, which plays the role of the ``mass": $E_{kin}\sim\frac{1}{2}\mathcal{M}\dot{q}^2$.

\section{Finite-temperature density}

As in any statistical theory, one first must determine which sort of ensemble properly describes the system. Nuclei are finite systems which have a conserved number of particles; however, HFB theory explicitly breaks particle number symmetry. In principle, we should use a microcanonical ensemble to describe a nucleus as a closed, isolated system, but that turns out to be challenging to solve because it requires a full knowledge of the eigenspectrum of the nucleus. Based on the particle non-conserving nature of HFB theory, we instead describe our system using the grand canonical ensemble.

The formalism to do this is shown in~\cite{Goodman1981}. In the end, finite-temperature HFB looks almost identical to standard zero-temperature HFB, except the density in the quasiparticle basis is replaced by
\begin{equation}
\mathcal{R} =
\left(\begin{array}{cc}
0 & 0 \\
0 & 1
\end{array}\right)
\rightarrow
\left(\begin{array}{cc}
f & 0 \\
0 & 1-f
\end{array}\right),\label{eq:FTdensity}
\end{equation}

\noindent with the Fermi factor $f$ given by $f_\mu=\frac{1}{1+e^{\beta E_\mu}}$.

\section{ATDHFB equations}

With the adiabatic assumption in place, we can write the density as an expansion around some time-even zeroth-order density:
\begin{align}
\mathcal{R}(t) 
&= e^{i\chi(t)}\mathcal{R}_0(t)e^{-i\chi(t)} \\
&= \mathcal{R}_0 + \mathcal{R}_1 + \mathcal{R}_2 + \dots,
\end{align}

\noindent where $\chi$ is assumed to be ``small" (which is explained more rigorously in~\cite{Baranger1978}) and
\begin{align}\label{eqn:densities1}
\mathcal{R}_1 &= i\left[\chi, \mathcal{R}_0\right], \\
\label{eqn:densities2}\mathcal{R}_2 &= \frac{1}{2}\left[\left[\chi, \mathcal{R}_0\right], \chi\right] .
\end{align}

The HFB matrix, being a function of $\mathcal{R}$, is likewise expanded:
\begin{equation}
\mathcal{H} = \mathcal{H}_0 + \mathcal{H}_1 + \mathcal{H}_2 + \dots,
\end{equation}

\noindent and $\mathcal{R}$ and $\mathcal{H}$ are then substituted into the TDHFB equation~\eqref{eq:TDHFB}. Gathering terms in powers of $\chi$ yields
\begin{align}
i\hbar\mathcal{\dot{R}}_0 &= \left[\mathcal{H}_0, \mathcal{R}_1\right] + \left[\mathcal{H}_1, \mathcal{R}_0\right], \label{eq:ATDHFB1}\\
i\hbar\mathcal{\dot{R}}_1 &= \left[\mathcal{H}_0, \mathcal{R}_0\right] + \left[\mathcal{H}_0, \mathcal{R}_2\right]
+ \left[\mathcal{H}_1, \mathcal{R}_1\right] + \left[\mathcal{H}_2, \mathcal{R}_0\right].\label{eq:ATDHFB2}
\end{align}

\noindent These two equations are the ATDHFB equations. They can be solved self-consistently to find both $\chi$ and $\mathcal{R}_0$; however, this is rarely done in practice. A more common approach is to exploit the fact that solutions to the ATDHFB equations are (by design) \textit{close} to true HFB solutions. One can then take an HFB solution and compute its time derivative by the first ATDHFB equation~\eqref{eq:ATDHFB1} to get ATDHFB-like behavior instead of a true solution to the ATDHFB equations.

Working in the HFB quasiparticle basis, in which
\begin{equation}
\mathcal{H}_0 =
\left(\begin{array}{cc}
E & 0 \\
0 & -E
\end{array}\right),
\qquad
\mathcal{R}_0 =
\left(\begin{array}{cc}
f & 0 \\
0 & 1-f
\end{array}\right),
\end{equation}

\noindent one finds by combining equations~\eqref{eqn:densities1} and~\eqref{eq:ATDHFB1} that the perturbation parameter $\chi$ is related to $\mathcal{\dot{R}}_0$, which was assumed to be small by the adiabatic hypothesis:
\begin{align}\label{eqn:chi-rdot_uncranked}
\begin{aligned}
\hbar \dot{\mathcal{R}}_{(0),ab}^{11} &= (E_a-E_b)(f_b-f_a)\chi_{ab}^{11} + (f_b-f_a)\mathcal{H}^{11}_{(1),ab}, \\
\hbar \dot{\mathcal{R}}_{(0),ab}^{12} &= (E_a+E_b)\left(1-(f_a+f_b)\right)\chi_{ab}^{12} + \left(1-(f_a+f_b)\right)\mathcal{H}^{12}_{(1),ab}, \\
\hbar \dot{\mathcal{R}}_{(0),ab}^{21} &= (E_a+E_b)\left(1-(f_a+f_b)\right)\chi_{ab}^{21} - \left(1-(f_a+f_b)\right)\mathcal{H}^{21}_{(1),ab}, \\
\hbar \dot{\mathcal{R}}_{(0),ab}^{22} &= (E_a-E_b)(f_b-f_a)\chi_{ab}^{22} - (f_b-f_a)\mathcal{H}^{22}_{(1),ab},
\end{aligned}
\end{align}

\noindent where the superscripts refer to $\frac{N}{2}\times\frac{N}{2}$ subblocks of the full $N\times N$ matrix. It is common (the so-called ``cranking approximation") to assume that changes in the density have approximately no effect on the mean field, in which case these relations reduce to
\begin{align}\label{eqn:chi-rdot}
\begin{aligned}
\hbar \dot{\mathcal{R}}_{(0),ab}^{11} &= (E_a-E_b)(f_b-f_a)\chi_{ab}^{11}, \\
\hbar \dot{\mathcal{R}}_{(0),ab}^{12} &= (E_a+E_b)\left(1-(f_a+f_b)\right)\chi_{ab}^{12}, \\
\hbar \dot{\mathcal{R}}_{(0),ab}^{21} &= (E_a+E_b)\left(1-(f_a+f_b)\right)\chi_{ab}^{21}, \\
\hbar \dot{\mathcal{R}}_{(0),ab}^{22} &= (E_a-E_b)(f_b-f_a)\chi_{ab}^{22}.
\end{aligned}
\end{align}

Finally, by expanding the total HFB energy up to second-order in $\chi$, the total energy of the system is found to be
\begin{equation}
E(\mathcal{R}) = E_{HFB} + \frac{1}{2}\mathrm{Tr}\left(\mathcal{H}_0\mathcal{R}_1\right) + \frac{1}{2}\mathrm{Tr}\left(\mathcal{H}_0\mathcal{R}_2\right) + \frac{1}{4}\mathrm{Tr}\left(\mathcal{H}_1\mathcal{R}_1\right).
\end{equation}

\noindent The ``kinetic energy" of the system is given by the latter two terms, which are both second order in $\chi$:
\begin{equation}\label{eqn:kinetic_energy}
E_{kin}(\mathcal{R}) = \frac{1}{2}\mathrm{Tr}\left(\mathcal{H}_0\mathcal{R}_2\right) + \frac{1}{4}\mathrm{Tr}\left(\mathcal{H}_1\mathcal{R}_1\right).
\end{equation}


\section{Finite-temperature ATDHFB inertia}
If we solve the ATDHFB equations~\eqref{eq:ATDHFB1} and~\eqref{eq:ATDHFB2} and then substitute equations~\eqref{eqn:chi-rdot} and~\eqref{eqn:RdotQdot} into the resulting expression for the kinetic energy~\eqref{eqn:kinetic_energy}, we end up with the following result:
\begin{equation}
	E_{kin} \approx \frac{1}{2}\sum_{\mu\nu}\dot{q}_\mu\dot{q}_\nu\mathcal{M}_{\mu\nu},
\end{equation}

\noindent where the collective inertia $\mathcal{M}$ is given by
\begin{align}\label{crankedFiniteTempInertia}
	\begin{aligned}
	\mathcal{M}_{\mu\nu} =  \frac{\hbar^2}{2}&\left[\frac{1}{(E_a-E_b)(f_b-f_a)}\left(\frac{\partial\mathcal{R}^{11}_{(0),ab}}{\partial q_\mu}\frac{\partial\mathcal{R}^{11}_{(0),ba}}{\partial q_\nu}+\frac{\partial\mathcal{R}^{22}_{(0),ab}}{\partial q_\mu}\frac{\partial\mathcal{R}^{22}_{(0),ba}}{\partial q_\mu}\right)\right.+ \\
	&\left.\frac{1}{(E_a+E_b)(1-f_a-f_b)}\left(\frac{\partial\mathcal{R}^{21}_{(0),ab}}{\partial q_\mu}\frac{\partial\mathcal{R}^{12}_{(0),ba}}{\partial q_\nu}+\frac{\partial\mathcal{R}^{12}_{(0),ab}}{\partial q_\mu}\frac{\partial\mathcal{R}^{21}_{(0),ba}}{\partial q_\mu}\right)\right].
	\end{aligned}
\end{align}

\noindent In the zero temperature case, the first piece (involving $\mathcal{R}^{11}$ and $\mathcal{R}^{22}$) goes to zero.

\section{Numerical implementation}

Numerical implementations of the finite-temperature ATDHFB inertia may encounter some challenges. First, as discussed in Section~\ref{sect:M_numerical} we know there is an ideal range of values $\delta q$ to use when computing derivatives of $\mathcal{R}$ using finite differences. Too large, and the derivative is artificially-smoothed out; too small, and densities of adjacent points can become indistinguishable (dependent on the HFB convergence parameter). Additionally, there may be problems when dealing with small, non-zero temperatures (or, more generally, when $f_b-f_a$ is small).
%
%Densities computed in HFODD are accurate to within about 10\% of the convergence parameter. So this means that we can't probably trust our densities to any precision greater than, roughly speaking, the convergence parameter.
%
%So what does this mean for us? Specifically, what does this mean for our differentiation by finite differences with various sizes $\delta q$, and how does this change when we start dividing by our $f$s?

We might consider introducing a cutoff on the difference $f_b-f_a$. If we make our cutoff too tight, we start cutting off physics (the tail of the Fermi distribution). If we leave our cutoff too small, we start dividing by numbers that are smaller than the noise in the density matrix which will lead to the inertia blowing up.

As an example, suppose we take a finite-difference spacing $\delta q = 10^{-3}$ and set our HFB convergence parameter to $10^{-7}$. Roughly-speaking, then, we would expect that we can set a cutoff for $f_b-f_a$ of something around $10^{-4}$ or so and still obtain reasonable results.

This was done for a $^{240}$Pu calculation ($Q_{20}=77$b, $Q_{22}=1$b, chosen at random from the $^{240}$Pu PES) with a few different cutoff values. First the exact T=0 inertia was computed. Then the inertia was computed again using the same T=0 densities, but with a fake temperature T=0.05 MeV. The results, shown in the following table, indicate the presence of an ideal cutoff parameter that is in the neighborhood of our prediction.

\begin{tabular}{|ccc|}
	\hline Convergence: & $10^{-7}$ & MeV \\ 
	$\delta q$: & $10^{-3}$ & b \\ 
	Actual T=0 inertia: & 1.585632$\times10^{-2}$ & $\frac{\hbar^2}{\mathrm{MeV \cdot b}^2}$ \\ \hline
	\textbf{Cutoff} & \textbf{Inertia} $\left(\frac{\hbar^2}{\mathrm{MeV \cdot b}^2}\right)$ & \textbf{\% error} \\ \hline
	$10^{-4}$ & 1.585205$\times10^{-2}$ & -2.692933$\times10^{-2}$ \\
	$10^{-5}$ & 1.585622$\times10^{-2}$ & -6.306634$\times10^{-4}$ \\
	$10^{-6}$ & 1.585697$\times10^{-2}$ & 4.099312$\times10^{-3}$ \\
	$10^{-7}$ & 1.587771$\times10^{-2}$ & 1.348989$\times10^{-1}$ \\
	$10^{-8}$ & 1.612200$\times10^{-2}$ & 1.675546 \\ \hline
\end{tabular} 
